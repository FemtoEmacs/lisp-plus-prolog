\documentclass[a4paper,12pt]{book}
\usepackage[english]{babel}
\usepackage[utf8]{inputenc}
\usepackage{graphicx}
\usepackage{menukeys}
\usepackage{amsmath}
\usepackage{amssymb}
\usepackage[T1]{fontenc}
\usepackage{upquote}
\usepackage{wrapfig}
\usepackage{makeidx}
\usepackage{tikz, pgf}
\usepackage{listing}
\usepackage{fancyvrb}


\usetikzlibrary{backgrounds, positioning, %% Draw Y-chart
snakes, fit,                            %% Draw Y-chart
3d,  arrows, automata, shapes.gates.logic.US,
shapes.gates.logic.IEC, calc, 
circuits.logic.US}
\usepackage{circuitikz}

\newsavebox{\fmbox}
\newenvironment{fmpage}[1]
{\begin{lrbox}{\fmbox}\begin{minipage}{#1}}
{\end{minipage}\end{lrbox}\fbox{\usebox{\fmbox}}}

\addto\captionsenglish{\renewcommand{\figurename}{Listing}}

\makeindex

\title{Prolog embedded in Lisp\\
{\normalsize Scheme version}\\
{\normalsize (Content not (yet) checked for grammar and spelling errors)}}
\author{Vernon Sipple \and ed500ac \and Marcus \and Lucas}
\date{}

\begin{document}
\maketitle
\thispagestyle{empty}

\mainmatter

\chapter{Emacs commands}
Emacs is the text editor that Lisp programmers prefer.
To give you a head start, this chapter lists the basic
commands you need to edit a file.

In the following cheat sheet, \verb|Spc| denotes
the space bar, \verb|C-| is the \keys{Ctrl} prefix, \verb|M-| represents 
the  \keys{Alt} prefix and  $\kappa$ can be any key. 
\index{Emacs!Ctrl prefix}\index{Emacs!Alt prefix}

\index{Emacs!save file}\index{Emacs!exit editor}
\index{Emacs!seach}\index{Emacs!undo}
\paragraph{Basic commands.} \verb|C-|$\kappa$ -- Press and release \keys{Ctrl}
and \keys{$\kappa$} simultaneously\index{Emacs!kill line}
\begin{quote}
	\verb|C-s| then type some text, and press \verb|C-s| again 
	-- search a text.\\
	\verb|C-r| then type some text, and press \verb|C-r| again 
	-- reverse search.\\
	\verb|C-k| -- kill a line.\\
	\verb|C-h| -- backspace.\\
	\verb|C-d| -- delete char forward.\\ 
	\verb|C-Spc| then move the cursor -- select a region.\\
	\verb|M-w| -- save selected region into the kill ring. \\
	\verb|C-w| -- kill region.\\
	\verb|C-y| -- insert killed or saved region.\\
	\verb|C-g| -- cancel minibuffer reading.\\
	\verb|C-a| -- go to beginning of line.\\
	\verb|C-e| -- go to end of line.\\
	\verb|C-b| -- move backward one character.\\
	\verb|C-f| -- move forward one character.\\
	\verb|C-n| -- move cursor to the next line.\\
	\verb|C-p| -- move cursor to the previous line.\\
	\verb|C-l| -- refresh screen.\\
	\verb|C-u| -- undo. \\
	\verb|C-v| -- forward page.\\
	\keys{INS} -- toggle overwrite mode.\\
	\keys{$\leftarrow$} \keys{$\rightarrow$} \keys{$\uparrow$} \keys{$\downarrow$} -- arrows move cursor.\\
\end{quote}
\paragraph{Control-x commands.} \verb|C-x C-|$\kappa$ -- Keep \keys{Ctrl}
down and press \keys{x} and  \keys{$\kappa$}
\begin{quote}
	\verb|C-x C-f| -- open a file into a new buffer.\\
	\verb|C-x C-w| -- write file with new name.\\
	\verb|C-x C-s| -- save file.\\
	\verb|C-x C-c| -- exit Emacs.\\
	\verb|C-x i| -- insert file at the cursor.\\
	\verb|C-h b| -- lists all key strokes.\\
\end{quote}



\paragraph{Window commands.} One can have more
than one window on the screen. Below, you will
find commands that deal with this situation.
\begin{itemize}\index{Emacs!split window}
		\index{Emacs!other window}\index{Emacs!switch buffer}
		\index{Emacs!close window}
	\item \verb|C-x |$\kappa$ -- Press and release \keys{Ctrl}
		and \keys{x} together, then press \keys{$\kappa$} 
		\begin{quote}
			\verb|C-x b| -- next buffer.\\
			\verb|C-x C-b| -- list buffers.\\
			\verb|C-x k| -- kill current buffer.\\
			\verb|C-x =| -- code for char under the cursor.\\
			\verb|C-x 2| -- split window into cursor window and other window.\\
			\verb|C-x o| -- jump to the other window.\\
			\verb|C-x 1| -- close the other window.\\
			\verb|C-x 0| -- close the cursor window.
		\end{quote}
	\item \verb|Esc| $\kappa$ -- Press and release the \keys{Esc} key,
		then press the \keys{$\kappa$} key.
		\begin{quote}
			\verb|Esc >| -- go to the end of buffer.\\
			\verb|Esc <| -- go to the beginning of buffer.\\ 
			\verb|Esc f| -- word forward.\\
			\verb|Esc b| -- word backward.\\
		\end{quote}
	\item \verb|M|-$\kappa$ -- Keep the \keys{Alt} key
		down and press the \keys{$\kappa$} key.
		\begin{quote}
			\verb|M-b| -- move backward one word.\\
			\verb|M-f| -- move forward one word.\\
			\verb|M-g| -- go to the line given in the minibuffer.\\
			\verb|M-n| -- activate the next buffer.\\
			\verb|M-r| -- query replace.
		\end{quote}
\end{itemize}

\paragraph{Query replace.} If you press \keys{Esc}\keys{\%},
the computer enters into the query replace mode.
First, Emacs prompts for the text snippet S
that you want to replace. Then it prompts for
the replacement R. When you type the R text and
press the \keys{Enter} key, the cursor jumps to
the first occurence of S, and Emacs asks whether
you want to replace it. If the answer is \verb|y|,
Emacs will replace S with \verb|R| and jump to
the next occurence. If the answer is \verb|n|,
Emacs will jump to the next occurrence of S without
performing the replacement.

\paragraph{Go to line.} When you try to compile
code containing errors, the compiler
usually reports the line number where the error occurred.
If you press \verb|C-u 3| \keys{Esc}\keys{g}\keys{Esc}\keys{g},
the cursor jumps to the line where the error occurred,
for instance, to line 3.

In order to know the position of the cursor, press
the \verb|C-x =| command, and Emacs gives information
concerning the character under the cursor, the corresponding
code, the line, and the character position in the text.

\paragraph{List bindings.} If you press
the \verb|C-h b| command, the computer
lists all Emacs commands. Then you
can check whether I forgot one or other keybinding in this
tutorial. In order to issue this command, keep
the \keys{Ctrl} key down and press \keys{x}, then release
both keys and press the \keys{?} key.

\paragraph{Nia Vardalos.}
In order to get acquainted with editing and scripting,
let us accompany the Canadian programmer Nia Vardalos,
while she explores Emacs. Disclaimer: Nia, the coder,
is not related to Nia, the actress.

When text is needed for carrying out a command,
it is read from the minibuffer, a one line
window at the bottom of the screen. For instance, 
if Nia presses \verb|C-s| for finding a text, 
the text is read from the minibuffer, and while Nia is
still typing, Emacs starts the search. When she finds
what she is looking for, Nia strikes the \keys{Enter}
key to stop the interactive search. Nia can press
\verb|C-s| again to find other instances of the text. 

To transport a region from one place to another,
Nia presses \verb|C-Spc| to start
the selection process and move the cursor to select
a region.  Then she presses \verb|C-w| to kill her
selection. Finally, she moves the cursor to the
insertion place, and presses the \verb|C-y| shortcut.
To copy a region, Nia presses \verb|C-Spc| and moves
the cursor to select the region. Then she presses
\verb|M-w| to save the selection into the kill ring. Finally,
she takes the cursor to the destination, where the copy is to
be inserted and issues the \verb|C-y| command.

Nia noticed that there are two equivalent ways to
issue an \verb|M-|$\kappa$ command. She can
press and release the \keys{Esc} key, then
strike the  $\kappa$ key. Alternatively, she
can keep the \keys{Alt} key down and press
the \keys{$\kappa$} key.

\paragraph{Calculations with Lisp.} Emacs
offers a Lisp dialect to perform calculations
and text processing. Let us assume that you
want to find out how many students graduate from
medical schools in California. You type the Lisp
addition command, then press \verb|C-x C-e| with
the cursor in front of it:
\begin{quote}
	\verb|> (+ 190 180 170 160 120 100 100 90)|\keys{C-x C-e}
\end{quote}
In the above expression, the first thing that follows
the open parenthesis is the \verb|+| sum
operation identifier. After the name of
the operation, there is a list of
the arguments to be added together.
Emacs shows the result of the addition as soon as
Nia presses the \keys{C-x C-e} command:
\begin{quote}
	\verb|1110|
\end{quote}
The rule that worked for  the $\verb|(+ | x_1\;x_2\ldots\verb|)|$
sum operation also works for the  other arithmetic function too.
\begin{itemize}
	\item{Multiplication} is performed
		by the \verb|*| operation.
		\begin{quote}
			\verb|> (* 1 2 3 4 5)|\keys{C-x C-e}\\
			120
		\end{quote}
	\item{Division} is obtained through the \verb|/| operation.
		\begin{quote}
			\verb|> (* 1 2 4)|\keys{C-x C-e}\\
			0.125
		\end{quote}
		The above operation performed the chain
		division \verb|1/2/3|.
	\item{Subtraction} has the syntax shown below.
		\begin{quote}
			\verb|> (- 1 2 4)|\keys{C-x C-e}\\
			-5
		\end{quote}
	\item Mixed operations. One can nest subexpressions
		within an outer expression. For instance,
		to get the average student number graduating
		from medical school in California, one can
		calculate the expression:
		\begin{quote}
			\verb|> (/ (+ 190 180 170 160 120 100 100 90) 8)|\keys{C-x C-e}
		\end{quote}

\end{itemize}

\paragraph{Command execution.} When you press 
shortcut keys, Emacs calls a command written in
C or in Lisp. You can get a complete list of all
Emacs commands by pressing the \verb|C-x ?| shortcut.

There are commands that are not associated to
shortcut keys. For instance, the \verb|scheme-mode|
function tells Emacs that you are editing Lisp code. 
In order to issue such a command, keep the \keys{Alt}
key down and press the \keys{~x~} key. The \verb|M-x|
prompt will be placed on the minibuffer. Type 
\verb|scheme-mode| in front of the prompt:
\begin{quote}
	\verb|M-x scheme-mode|
\end{quote}
When typing a command at the \verb|M-x| prompt, you
often don't remember the complete name of the operation.
In this case, type press the \keys{Tab} key, and Emacs will
list the options available to complete the name.

\begin{verbatim}
backspace        backspace
C-a              move-beginning-of-line
C-b              backward-char
C-d              delete-char
C-e              move-end-of-line
C-f              forward-char
C-h              help-command
C-k              kill-line
C-l              recenter-top-bottom
C-n              next-line
C-p              previous-line
C-r              isearch-backward
C-space          set-mark
C-s              isearch-forward
C-/              undo
C-v              scroll-up-command
C-w              kill-region
C-x 1            delete-other-windows
C-x 2            split-window-below
C-x C-b          list-buffers
C-x C-c          save-buffers-kill-terminal
C-x C-f          find-file
C-x C-s          save-buffer
C-x =            cursor-position
C-x C-w          write-file
C-x i            insert-file
C-x k            kill-buffer
C-x o            other-window
C-y              yank
down             next-line
esc a            apropos
esc B            backward-word
esc b            backward-word
esc <            beginning-of-buffer
esc d            kill-line
esc >            end-of-buffer
esc esc          show-version
esc F            forward-word
esc f            forward-word
esc G            goto-line
esc g            goto-line
esc i            yank
esc k            kill-region
esc l            list-bindings
esc m            set-mark
esc n            next-buffer
esc o            delete-other-windows
right            forward-character
up               previous-line
\end{verbatim}

\chapter{Installation}  
In order to program in Prolog, you will need
to install three programs:\index{wamcompiler}
\begin{enumerate}
	\item Emacs -- a text editor\index{wamcompiler!Emacs}
	\item The Racket Language (friendlier), or Chez Scheme
		(faster) \index{cxprolog!scheme}
	\item A Prolog compiler that works in tandem with Lisp,
		such as \verb|cxprolog| that links Lisp with a robust
		implementation of Prolog, or a Lisp based Prolog, such
		as  the \verb|wamcompiler.lisp| virtual machine.
\end{enumerate}

However, you must learn many things before actually being
able to make these three pieces of software work together. 
Unless you are well versed in programming, I suggest that
you call a geek who has majored in Computer Science to perform
the installations for you.

Installation of Emacs is quite easy. You will find binary
distributions for practically any machine. Even so, you
will need help to configure the editor and perform the
setup of useful plug-ins, such as slime, which will help
you with Lisp programming.

Installation of Racket is also easy for the same reasons
as Emacs: binaries are available. Installation of Chez
Scheme is somewhat harder, since you need to build it
from source. Search the Internet for the Racket or Ches
Scheme page  and follow the instructions with the help
of the Computer Science major.\index{scheme!github}

Besides Prolog, this book will deal with Scheme Lisp
and shell script, thus if you go through the whole
text, you will not need help from the Computer Science
major anymore.

Them main message that the authors of this book want to
convey is that Lisp does not become obsolete, since it
has solid mathematical foundations, and mathematics does
not become obsolete.

\section{Ready}
If you installed Emacs and scheme, you are now ready for
action!

\includegraphics{figs-prefix/readyforaction.jpg}

It seems that people prefer money to sex.
After all, almost everybody says no to
sex on one occasion or another.
But I have never seen a single
person refusing money. Therefore, let us
start this tutorial talking about money.

If one wants to calculate a running
total of bank deposits, she will make
a column of numbers and perform the
addition.

\index{Arithmetic!Elemetary School}  
\begin{wrapfigure}[10]{i}{5cm}
	\renewcommand\figurename{Fig.}
	\includegraphics[scale=0.5]{figs-prefix/firemen.png}
	\caption{Running total}
\end{wrapfigure}
What I mean to say is that, if you
need to perform the addition
$8 + 26 + 85 + 3$ with pencil and paper,
you will probably stack the numbers the
same way you did before taking pre-algebra
classes in highschool:
\begin{quote}
	\begin{tabular}{p{0.5cm}p{1cm}}
		+ &\verb|  8|\\
		&\verb| 26|\\
		&\verb| 85|\\
		&\verb|  3|\\
		\hline
		&\verb|122|
	\end{tabular}
\end{quote}
Subtraction is not treated differently:
\begin{quote}
	\begin{tabular}{p{0.5cm}p{1.5cm}}
		\Large\bf -&   358\\
		&   216\\
		\hline
		& 142
	\end{tabular}
\end{quote}

This chapter contains an introduction to
the Cambridge prefix notation, 
which is only slightly different from that
learned in elementary school: The operation
and its arguments are put between parentheses.
In doing so, one does not need to draw a line
under the bottom number, as you can see below:
\begin{quote}
	\begin{tabular}{p{0.5cm}p{1cm}}
		(+ &\verb|  8|\\
		&\verb| 26|\\
		&\verb| 85|\\
		&\verb|  3|)\\
		&\verb|122|
	\end{tabular}
\end{quote}
The Cambridge prefix notation can be applied
to any operation, not only to the four arithmetic
functions. 

\section{Cambridge prefix notation}
\index{Prefix notation}
\index{Prefix notation!Cambridge}
Let us summarize what we have learned
until now. In pre-algebra, students
learn to place arithmetic operators (+, -, × and ÷)
between their operands; e.g. 347+45.
However, when doing additions and subtractions
on paper, they stack the operands.

\includegraphics{figs-prefix/stackshark.jpg}


Lisp programmers put operation and operands
between parentheses. The right parenthesis
separate the operands from the result,
instead of drawing a line under the last operand.

\includegraphics{figs-prefix/neatsum.jpg}

\section{Pictures}
Here is the story of a Texan who went on
vacation to a beach in Mexico. While he was
freely dallying with the local beauties,
unbeknowest to him a blackmailer took some
rather incriminating photos.
After a week long gallivanting, the Texan
returns to his ranch in a small town near Austin.
Arriving at his door shortly after is the blackmailer
full of bad intentions.

Unaware of any malice, the Texan allows the so
called photographer to enter and sit in front
of his desk. Without delay, the blackmailer spread
out a number of photos on the desk, along with his
demands: “For the photo in front of the hotel,
that will cost you \$~25320.00. Well, the one
on the beach that's got to be \$~56750.00.
Finally, this definitively I can't let
go for less than \$~136000.00.”

Once finished with his presentation,
the blackmailer snaps up the photos,
and looks to the Texan with a sinister
grin, awaiting his reply.

Delighted with the selection of pics,
the Texan in an elated voice says:
“I thought I would have no recollection
of my wonderful time. I want 3 copies
of the hotel shot, half a dozen of the beach.
And the last one, I need two copies for myself,
and please, send one to my ex-wife.
Make sure you leave me your contact
details; I might need some more.

\paragraph{Mixed calculations.}
In order to calculate how much the
Texan must pay his supposed blackmailer,
his bookkeeper needs to perform the
following operations:
$$3\times 25320+6\times 56750 + 2\times 136000+136000$$
It should be remembered that multiplications
within an expression take priority over additions
and subtractions. The bookkeeper must therefore
calculate the first two products $3\times 25320$
and $6\times 56750$, before performing the first
addition. In the Cambridge prefix notation,
the internal parentheses pass priority over
to the multiplications.

The Texan's bookkeeper started Emacs
in order to perform the calculations.
The text editor creates a memory
buffer that mirrors the file contents.
By convention, the file and the buffer
have the same name.

Initially, Emacs will place the bookkeeper
on a *scratch* buffer. The bookkeeper issues
the \verb|M-x| command by keeping the \keys{Alt}
key down and pressing the \verb|x| key.
Emacs will put the bookkeeper in the minibuffer
with the \verb|M-x| prompt. Then, the bookkeeper
will type the \verb|shell| command, which we
will study in one of the future chapters.
Here is how to start the shell:
\begin{quote}
	\verb|M-x shell| \keys{Enter}
\end{quote}
The above command will create a kind of chat
box between the bookkeeper and the computer.
The bookkeeper types the \verb|ros run| command
to start Lisp.  Listing~\ref{texan:photos} shows
what an interaction with shell and Racket looks
like. After typing an expression, while obeying the
Cambridge prefix notation syntax,
the bookkeeper moves the cursor to the front
of the most external right parenthesis.
To perform the calculation, she presses
the \keys{Enter} key. Then, the sbcl Read Eval
Print Loop transforms the command to machine
language, evaluates it, and inserts the result into
the buffer.

To save the buffer, the bookkeeper
keeps the \keys{Ctrl} key down, and presses
the \keys{~x~} and \keys{~s~} keys in sequence.
To exit the editor, she maintains the \keys{Ctrl}
key pressed and hits the \keys{~x~} and \keys{~c~}
keys one after the other. If there are unsaved files,
the editor warns that modified buffers exist,
and requires confirmation before quitting. But wait,
if you are shadowing the actions of the bookkeeper,
don't leave the editor yet. Let us test the Scheme
compiler before doing so. Below, the accountant is
running Chez Scheme.

\begin{figure}[!h]
	\begin{fmpage}{0.8\textwidth}
		\begin{verbatim}
		› scheme
		Chez Scheme Version 9.5.3
		Copyright 1984-2019 Cisco Systems, Inc.
		\end{verbatim}
		\verb|> (+  (* 3 25320)|\\
		\verb|      (* 6 56750)|\\
		\verb|      (* 2 136000)|\\
		\verb|    136000 )|\keys{Enter}\\ 
		824460
	\end{fmpage}
	\begin{fmpage}{0.8\textwidth}
		\verb| |
	\end{fmpage}
	\caption{Chat with Lisp}
	\label{texan:photos}
\end{figure}

The Texan's accountant decided to check whether
Scheme is working on her machine. Therefore,
she calls \verb|racket| or \verb|chez scheme| and
starts a Prolog Read Eval Print Loop, or repl for
short.
\begin{quote}
	\begin{verbatim}
	› scheme
	Chez Scheme Version 9.5.3
	> (load "prolog.scm")
	> (logic "cxprolog")
	CxProlog version 0.98.2 [development]

	[main] ?- X is 3*25320 + 6*56750+ 2* 136000 + 136000.
	X = 824460

	[main] ?- halt.
	% CxProlog halted.
	0
	> (exit)
	\end{verbatim}
\end{quote}

The Lisp prompt is usually the \verb|>| symbol,
therefore be careful not to confuse the prompt
with the {\em greater-than} \verb|>| character.
The Prolog prompt is the \verb|?-| symbol.

\section{Time value of money}
Suppose that you wanted to buy a \$ 100,000 red Ferrari,
and the forecourt salesperson in his eagerness to
close a deal gives you the following two payment options:  
\begin{itemize}
	\item Pay \$ 100,000 now {\bf or}
	\item pay the full \$ 100,000 after a three year grace period.
\end{itemize}

I am sure that you would choose to pay the \$ 100,000 after
three years of grace has finished, although you have the
money in a savings account waiting for a business
opportunity. But why is this? After all, you will need
to pay the debt one way or the other. However, if you
keep the money in your power during the grace period,
you can earn a few months of fuel from the interest.

You may not know for sure how much interest you
will earn in three years, but since the salesperson
is not charging you for deferring the payment,
whatever you gain is yours to keep.

Unfortunately for you, the above scenario would
more than probably not happen in real life.
The right to delay payment until some future
date is a merchandise with a price tag,
which {\em is called interest by those who
think it lawful, and usury by those who
do not} (William Blackstone's Commentaries
on the Laws of England). Therefore, unless
the salesperson is your favorite aunt,
the actual offer is like jumping into a tank
full of sharks as in the classic James Bond films.
It requires a little forethought and understanding
of how interest works, before making a decision.
Here is a more realistic real estate sales offer:

\begin{itemize}
	\item \$ 100,000 now {\bf or}
	\item \$ 115,000 at the end of three years.
\end{itemize}

What to do when facing an increase in price
to cover postponement of payment? The best
policy is to ask your banker how much interest
she is willing to pay you over your
granted grace period.

Since the economy performance is far
from spectacular, your banker offers you
an interest rate of 2.5\%, compound annually.
She explains that compound interest arises
when interest is paid on both the principal
and also on any interest from past years.

\section{Future value}\index{Future Value}

\includegraphics{figs-prefix/piggy.jpg}

The value of money changes with time. Therefore,
the longer you keep control of your money,
the higher its value becomes, as it can earn
interest. Time Value of Money, or TVM for short,
is a concept that conveys the idea that money
available now is worth more than the same
amount in the future.

If you have \$ 100,000.00 in a savings account
now, that amount is called {\em present value},
since it is what your investment would give you,
if you were to spend it today.

Future value of an investment is the amount
you have today plus the interest that your
investment will bring at the end of a
specified period.

The relationship between the present value
and the future value is given by the
following expression:
\begin{equation}
	FV= PV\times (1+i)^n
	\label{eq:future-value}
\end{equation}
where $FV$ is the future value, $PV$ is the present
value, $i$ is the interest rate divided by 100,
and $n$ is the number of periods.

In the case of postponing the payment of a \$ 100,000.00 car
for 3 years, at an interest rate of 0.025, the future
value of the money would be 107,689.06; therefore,
I strongly recommend against postponing the payment.

\section{Compound interest}

Our Texan decides he needs a break. Thus he walks into
a New York City bank and asks for the loan officer.
He tells a story of how through his doctor's recommendation
he was taking it easy at his property in the south of
France for two whole years and for such an 
emergency he needs a \$ 10,000.00 loan.

The loan officer said that the interest was a compound 8\%
a year, but the bank would need some collateral for the loan.

“Well, I have a 60 year old car that I like very much.
Of course, I cannot take it with me to France.
Would you accept it as collateral?”

Unsure whether or not the old car was worth the
amount of the loan, the officer summons the bank manager.
The manager inspects the vehicle that was parked on the
street in front of the bank. After a close examination,
he gives a nod of approval: “It’s a Tucker Torpedo.
Give him the loan.”

The Texan quite willingly signed his heirloom over.
An employee then drove the old car into the bank’s
underground garage and parked it. From time to time,
the employee would go, and turn over the engine, to keep
the car in good running condition, and gave it an
occasional waxing just to maintain it in pristine condition.
Two years later the Texan returned, and asked how
much he owed the bank. The loan officer started Emacs and
Lisp, and calculated the total debt as \$ 11,664.00.

\paragraph{Interest Calculation.}
\index{Interest!Compound}
Let us follow the calculation of the value accumulated
in the first year of compound interest at an 8\% rate
over \$ 10,000.00. Enter formula~\ref{eq:future-value} in
the Lisp Read Eval Print Loop:
\begin{quote}
	\begin{verbatim}
	> (* (expt (+ 1.0 0.08) 2) 10000)
	11664.000000000002
	\end{verbatim}
\end{quote}
Observe that it was necessary to divide the
interest rate by 100, which produces 0.08.

\begin{figure}[!h]\index{define}
	\begin{fmpage}{0.8\textwidth}
		\begin{verbatim}
		;; File: fvalue.scm

		(define  (future-value pv i n)
   (* (expt (+ (/ i 100.0) 1) n) 
       pv)
       );;end define
		\end{verbatim}
	\end{fmpage}

	\begin{fmpage}{0.8\textwidth}
		\verb|> (load "fvalue.scm")|\keys{Enter}\\
		\verb|> (future-value 10000 8 2)|\keys{Enter}\\
		11644.00
	\end{fmpage}
	\caption{Future Value Program}
	\label{Texan:parking}
\end{figure}


It is possible to define an operation that
calculates formula~\ref{eq:future-value}
given the arguments \verb|fv|, \verb|i|
and \verb|n|. The definition is given
in listing~\ref{Texan:parking}.

After typing the definition shown in listing~\ref{Texan:parking},
you must issue the \verb|C-x C-s| command to save the buffer.
In order to do this, keep the \keys{Ctrl} key pressed down and
hit the \keys{~x~} and \keys{~s~} keys in sequence.

In order to load the program into Lisp, keep the \keys{Alt} down
and press the \verb|x| key. The prompt \verb|M-x| will appear
in the minibuffer; type \verb|shell| at this prompt, as shown below.\\

\begin{fmpage}{0.8\textwidth}
	\begin{verbatim}
	;; File: fvalue.scm

	(define  (future-value pv i n)
   (* (expt (+ (/ i 100.0) 1) n) 
       pv)
       );;end define
	\end{verbatim}
\end{fmpage}

\begin{fmpage}{0.8\textwidth}
	\verb|M-x shell|\keys{Enter}\\
\end{fmpage}

\verb||\\
When the shell starts, run the \verb|scheme| command
to launch Chez Scheme.
\begin{verbatim}
› scheme
Chez Scheme Version 9.5.3

> (load "fvalue.scm")
> (future-value 10000 8 2)
11664.000000000002
> (exit)   ;; Here is how to quit Scheme
\end{verbatim}

In listing~\ref{Texan:parking}, any text
between a semicolon and the end of
a line is considered a comment.
Therefore, \verb|;; File: fvalue.scm|
is a comment. Likewise, \verb|;;end define|
is a comment. Comments are ignored by
Lisp, and have the simple function of
helping users and programmers.

Now, let us see what Prolog adds to Common Lisp,
I mean, what Lisp programmers will gain by
having a Prolog compiler embedded into Common
Lisp. Here one can see a Prolog program for
calculating future values:
\index{Prolog!future value}
\begin{verbatim}
%% File: ?- consult('fvalue.pl')

period_of_time(2).
period_of_time(4).
period_of_time(8).
period_of_time(10).

futval(PV, I, N, FV) :- period_of_time(N),
		 FV is PV*(1+I/100)**N.
\end{verbatim}

\index{Prolog!comments}
In Prolog, comments are introduced by the \%
percent symbol, not by a semicolon, as in Lisp.
Variables are capitalized; thus, in the definition
of \verb|futval/4|, the symbols \verb|PV|, \verb|I|,
\verb|N|, and \verb|FV| are variables. Below you will
see an example of how to compile and execute
the \verb|'fvalue.pl'| program. The predicate
\verb|period_of_time/1| has four clauses. When
this predicate is called from \verb|futval/4| with the 
\verb|period_of_time(N)| pattern, it unifies \verb|N|
with the first pattern, which makes \verb|N= 2|.
To make a long story short, there are many
options for the period of time \verb|N|, and Prolog
starts with the first choice.

The \verb|FV| future value is calculated for this choice
of \verb|N|, and the result is shown to the user. If the
first result is satisfactory, the user types
\verb|yes|, and Prolog stops searching for
solutions. If the user presses the semicolon
key, Prolog backtracks to the \verb|period_of_time(N)|
predicate, and chooses \verb|N=4|, which is
the second option. By repeatedly pressing
the semicolon, the user is able to obtain all solutions.

\begin{verbatim}
› scheme
Chez Scheme Version 9.5.3
Copyright 1984-2019 Cisco Systems, Inc.

> (load "prolog.scm")
> (logic "-f fvalue.pl")
CxProlog version 0.98.2 [development]

[main] ?- futval(10000, 8, N, FV).
N = 2
FV = 11664 ? ;
N = 4
FV = 13604.8896 ? ;
N = 8
FV = 18509.3021 ? .
...
?- halt.
% CxProlog halted.
0
> (exit)
\end{verbatim}

\section{Mortgage}
A man with horns, legs and tail of a goat,
thick beard, snub nose and pointed ears entered
a real estate agency, and expressed his intention
of closing a deal. People fled in all directions,
thinking that Satan himself was paying them a visit.
Deardry seemed to be the only person who remained
calm. “What a ignorant, narrow minded, prejudiced
lot! I know you are not the devil. You are the
Great God Pan! What can I do for you?”

“I would like to buy a house in London.
I know that the city is made up of 32 Boroughs
and where I buy will make an enormous difference
to price, quality of life, and chances of increase
in capital value of the property.”

“Well, Sir, with our experience you can rest
assured that you will secure your ideal property.
Firstly we must decide on what type of property
fits your needs and where you want to be. You
may consider somewhere like Fulham, Chelsea,
Knightsbridge, Kensington or Mayfair.
I have properties in all these places. ”

“Chelsea! I like the name.”

“Chelsea is arguably one of the best residential
areas in London. It has benefited from it’s close
proximity to the west end of the city and is highly
sought after by overseas buyers looking to be located
in one of the most popular areas in central London.
At the moment, I can offer you a fantastic living
space in a four bedroomed flat situated behind
King's Road. The price is £10,500,000.”

“I don't have this kind of cash with me right now,
but can get it in a few months of working in a
circus. Meanwhile, can you arrange a mortgage for me? ”

“Yes, Yes I can help arrange mortgages in all 32 Boroughs
of London. Non-residents can have mortgages up to 70\% of
the purchase price. Do you have £3,150,000 pounds for the
down payment? In mortgage agreements, down payment is
the difference between the purchase price of a property
and the mortgage loan amount.”

“I know what down payment is. And yes, I can dispose
of £3,150,000 pounds.”

“A mortgage insurance is required for borrowers with
a down payment of less than 20\% of the home's purchase
price. That is not your case. Therefore, the balance of
the purchase price after the down payment is deducted
is the amount of your mortgage. Let us write a program
in Lisp to find out your estimated monthly payment.
The loan amount is £7,350,000 pounds. The interest rate
is 10\% a year. The length of the mortgage is 20 years.
That is the best I can do for you, I am afraid.
You are going to pay £97131.00 pounds a month for
ten years. After that, the balance of your debt
will be zero. But your visit is a surprise!”
The broker exclaimed. “ I never thought I would
ever see a true living god wanting to do property
deals in London.”

With that, Pan replied: “If property prices
were not so high, and interest rate so steep,
we would definitely be up for business more often.”

\includegraphics[scale=0.5]{figs-prefix/deuspan.jpg}

\paragraph{Loan Amortization.}
Amortization refers to the gradual reduction
of the loan principal through periodic payments.

\includegraphics{figs-prefix/agingshark.jpg}

Suppose you obtained a 20-year mortgage
for a \$ 100,000.00 principal at the interest
rate of 6\% a year. Calculate the monthly payments.

\paragraph{Amortization Formula.} Let $p$ be the
present value, $n$ the number of periods,
and $r$ be the
interest rate, i.e., the interest divided by 100.
In this case, the monthly payment for full
amortization is given by~\ref{eq:amortization}.
This formula is implemented in
listing~\ref{Pan:mortgage}.
\begin{equation}
	\frac{r\times p\times(1+r)^n}{(1+r)^n - 1}
	\label{eq:amortization}
\end{equation}

\begin{figure}[!h]\index{define}
	\begin{fmpage}{0.9\textwidth}
		\begin{verbatim}
		;; File: (load "pmt.scm")

		(define (pmt p  i n )
      (define (pmt-aux r rn)
	  (/ (* r p rn)
	     (- rn 1)))
  (pmt-aux (/ i 100.0) (expt (+ 1.0 (/ i 100)) n))
  );;end of define
		\end{verbatim}
	\end{fmpage}

	\begin{fmpage}{0.9\textwidth}
		\begin{verbatim}
		› scheme
		Chez Scheme Version 9.5.3

		> (load "scm-src/pmt.scm")
		> (pmt (* 10500000.00 0.7) (/ 10.0 12) (* 20 12))
		70929.09
		\end{verbatim}
	\end{fmpage}
	\caption{Mortgage for a God}
	\label{Pan:mortgage}
\end{figure}

The \verb|pmt| function, shown on listing~\ref{Pan:mortgage},
defines a \verb|pmt-aux| that has two parameters \verb|r| and
\verb|rn|, which the user does not need to express.
Lisp itself assigns \verb|(/ i 100.0)| to \verb|r| and
\verb|(expt (+ 1.0 (/ i 100)) n)| to \verb|rn|. Auxiliary
functions, such as \verb|pmt-aux|, give meaningful names
for variables that identify subexpressions, and thus make
the main expression easier to understand. There are situations,
when auxiliary provide initial values to parameters, relieving
the user from such a task.

Now, let us repeat the magic in Prolog. Here is the program:\\

\begin{fmpage}{0.9\textwidth}
	\begin{verbatim}
	%% File: (logic "-f pmt.pl")

	property(fulham, 6500000).
	property(chelsea, 10500000).
	property(mayfair, 9000000).
	property(kensington, 4000000).

	pmt(Local, Price, Down, I, NY, Payment) :-
     property(Local, Price),
     P is 0.7 * Price,
     Down is 0.3 * Price,
     R is I/1200,
     N is NY * 12,
     RN is (1 + R)** N,
     Payment is R * P * RN / (RN - 1).
	\end{verbatim}
\end{fmpage}

\verb||\\
The Prolog program will consult a data base of
property prices, calculate the down payment and
the monthly payments needed to be made, and present the
values to the appreciation of the client. By pressing the
semicolon key, the client can refuse an offer and
ask to see another option.

\begin{verbatim}
› scheme
Chez Scheme Version 9.5.3
Copyright 1984-2019 Cisco Systems, Inc.

> (load "prolog.scm")
> (logic "-f prolog-src/pmt.pl")
CxProlog version 0.98.2 [development]

[main] ?- pmt(Local, Price, Down, 10, 20, Payment).
Local = fulham
Price = 6500000
Down = 1950000
Payment = 43908.48485 ? ;
Local = chelsea
Price = 10500000
Down = 3150000
Payment = 70929.09091 ? ;
Local = mayfair
Price = 9000000
Down = 2700000
Payment = 60796.36364 ? .
...
?- halt.
% CxProlog halted.
0
> (exit)
\end{verbatim}

Prolog programs are executed by an inference
engine. This means that you need to run the
scheme incremental compiler from the folder
where the inference engine is located. If you
are using CxProlog, then the Prolog folder
contains the \verb|libcxprolog| file, say,
the \verb|sxprolog| folder. On the other hand,
the \verb|prolog-src/pmt.pl| source file is
maintained in a subdirectory of \verb|sxprolog|.
You need to provide a complete path from the place
where you are working to the \verb|pmt.pl| file,
otherwise the \verb|(logic "-f prolog-src/pmt")|
command will not be able to find it.

\section{Functions}\index{function!mathematics}
I am sure that you know what a function is. It is
possible that you do not know the mathematical
definition of function, but you have a feeling
for functions, acquired by the use of calculators
and computer programs or by attending one of
those Pre-Algebra courses.

A function has a functor, which is the name of the
function, and arguments,e.g., \verb|S is sin(X)| is
a function. Another function is \verb|M is mod(X,Y)|
that gives the remainder of the division of \verb|X|
by \verb|Y|.  When you want to use a function, you
substitute constant values for the variables, or arguments.
For instance, if you want to find the remainder
of 13 divided by 2, you can type
\begin{quote}
	\verb|?- X is mod(13,2).|
\end{quote}
in the Prolog read eval print loop to get the
answer. One could say that functions are maps
between possible values for the argument and
a value in the set of calculation results.

The domain is the set of possible values for the
argument, whereas the image is the set of calculation
results. In the case of the sinus function,
the elements of the domain are real numbers
between $-\pi/2$ and $\pi/2$ radians. There
is an important thing to remember: 
Mathematicians insist that a function must produce
only one value for a given argument. Therefore,
if a calculation produces more than one value,
it is not a function. For instance, $\sqrt{4}$
can be 2 or -2. Then the square root of a number
cannot be a function. 

What about functions of many arguments? For example,
the function \verb|max(X,Y)| takes two arguments,
and returns the greatest one. In this case, you can
consider that it has only one argument, that being a
tuple of values. Thus, the argument of \verb|max(5,2)|
is the \verb|(5,2)| pair. Mathematicians say that the
argument of such a function is the Cartesian product
$R\times R$. The Cartesian product was named after
René Descartes, a French philosopher and scientist.
In Latin, the family name of Descartes translates
as Cartesius, therefore, the adjective Cartesian.

There are functions whose functor is placed between
its arguments. This is the case of the arithmetic
operations, where one often writes \verb|5+7|, instead
of \verb|+(5,7)|. At this point, it is necessary to
praise Lisp, since the designers of this elegant language
chose a uniform syntax for all operations, including
application of functions. In Lisp, the \verb|sin(X)|
function is represented by \verb|(sin X)|, and 
addition is notated as \verb|(+ X Y Z ...)|. In other
words, all well formed formulas in Lisp are lists, where
the operator is the first element.

\section{Predicates}\index{predicates}
Predicates are functions whose domain is the set
\verb|{true,false}|. There are a few predicates
that are well known to any person who tried his/her
hand at programming, or even by a student who is
taking Pre-Algebra:
\begin{itemize}
	\item $X>Y$ is true if $X$ is greater than $Y$, false otherwise.
	\item $X<Y$ is true if $X$ is less than $Y$, false otherwise.
	8\item $X=Y$ is true if $X$ is equal to $Y$, false otherwise.
\end{itemize}
A predicate with one argument tells of an attribute or feature
of its argument. One can tell that this predicate acts like
an adjective. In Lisp, \verb|(not X)| is true if \verb|X| is
false, and \verb|false| otherwise. In other computer languages
there are predicates equivalent to this one.

In the story of the Texan, you have seen an example of a one slot predicate:
\begin{quote}
	\begin{verbatim}
	period_of_time(2).
	period_of_time(4).
	period_of_time(8).
	period_of_time(10).
	\end{verbatim}
\end{quote}
A predicate with more than one argument shows that there
exists a relationship between its arguments. In the case of
\verb|X=Y|, the relationship is {\em equal}.

Prolog predicates have a feature that one cannot see in
other computer languages: they not only check whether
constant arguments have a feature or obey a relationship,
but they actively search for values that satisfy a
set of requisites. For example, the \verb|futval(PV, 8, N, FV)|
searches for values of \verb|N| and \verb|FV| that satisfy
the definition of a future value.

%% Vernon stopped here on 17/12/2019
\section{Lists}
Every human and computer language
has syntax, i.e., rules for combining
components of sentences.
In general, a computer language
syntax is complex, and the programmer
must study it for years before
becoming proficient.

Lisp has the simplest syntax
of any language. All
Lisp constructions are expressed
in Cambridge prefix notation:
$\verb|(operation|\;x_1\;x_2\;x_3\ldots x_n\verb|)|$.
This means that all Lisp procedures  can be written 
as a pair of parentheses enclosing an operation
followed by its arguments. 

A linked list is a representation for
a \verb|xs| sequence  in such a way
that it is easy and efficient to
push a new object to the top of \verb|xs|.

The \verb|'(S P Q R)| list of letters
is prefixed by a quotation mark because
in Lisp lists and programs have the
same syntax: The Cambridge prefix notation.
The single quotation mark
tags the list so the computer will
take it at face-value, and thus will
not try to be evaluated.
\begin{quote}
	\begin{verbatim}
	> (define aList '(S P Q R))
	(S P Q R)
	> (cons 'Rome aList)
	(Rome S P Q R)
	> aList
	(S P Q R)
	\end{verbatim}
\end{quote}
Through the repeated application of the
\verb|cons| function, one can
build lists of any length
by adding new elements to
a core.
\begin{quote}
	\begin{verbatim}
	> (define xs '(S P Q R))
	(S P Q R)
	> (cons 5 (cons 4 (cons 3 (cons 2 (cons 1 xs)) ) ))
	(5 4 3 2 1 S P Q R)
	> xs
	(S P Q R)
	\end{verbatim}
\end{quote}
In the above example, you have noted
that repeated application of nested functions
cause right parentheses accumulate
at the tail of the expression.
It is good practice in programming
to group these bunches of right parentheses
in easily distinguishable patterns.

\begin{verbatim}
> (cons 'S (cons 'P (cons 'Q '(R)) ))  ;4 right parentheses
(1 2 3)
> (+ 51 (* 9 (+ 2 (- 3 (+ 1 2 3)) ) )) ;5 right parentheses
42 
> (+ 6 (* 9 (+ 2 (- 3 (+ 4 (- 3))) ))) ;6 right parentheses
42
\end{verbatim}

Another way to create easily recognizable patterns  is to
reorganize the expression in order to interrupt the sequence
of right parentheses:
\begin{quote}
	\begin{verbatim}
	> (+ 51 (* (+ 2 (- 3 (+ 1 2 3))) 9))
	42
	\end{verbatim}
\end{quote}

The head of a list is its first element. For instance, the head of
\verb|'(S P Q R)| is the \verb|'S| element. The tail is the sublist
that comes after the head. In the case of \verb|'(S P Q R)|, the tail
is \verb|'(P Q R)|. Lisp has two functions, \verb|(car xs)| and
\verb|(cdr xs)| that selects the head and the tail respectively.
\index{List!selector: car}
\index{List!selector: cdr}

As one can see below, it is possible to reach any element of a list
with a sequence of \verb|car| and \verb|cdr|.
\begin{verbatim}
> (define xs '(2 3 4 5 6.0))
(2 3 4 5 6.0)
> xs
(2 3 4 5 6.0)
> (car xs)
2
> (cdr xs)
(3 4 5 6.0)
> (car (cdr xs))
3
> (cdr (cdr xs))
(4 5 6.0)
> (car (cdr (cdr xs)))
4
> (car (cdr (cdr (cdr xs)) ))
5
> (car (cdr (cdr (cdr (cdr xs)) ) ))
6.0
\end{verbatim}

\section{Going through a list}\label{sec:average}
In listing~\ref{going-through-a-list},
the \verb|avg| function can be used to
calculate the average of a list of numbers.
It reaches all elements of the list
through successive applications
of  \verb|(cdr s)|.
The list elements are accumulated in
\verb|acc|,
while the \verb|n| parameter counts
the number of \verb|(cdr s)| applications.
When  \verb|s| becomes empty, the \verb|acc|
accumulator contains the sum of \verb|s|, and \verb|n|
contains the number of elements. The average
is given by \verb|(/ acc n)|. 

In order to fully understand the workings
of the \verb|avg| function, we should have
waited for the introduction to the \verb|cond-form|
given on page~\pageref{page:cond-form}.
The only reason for presenting
a complex definition like \verb|avg|
so early in this tutorial is
to show a grouping pattern of five
right parentheses. As was discussed
previously, when the number of close
parentheses is greater than 3, good
programmers distribute these into small
groups, so that an individual who is
trying to understand the program
can see that the expression is
proprely closed at a glance.

\pagebreak
\paragraph{For the impatient learner.}
However, for the sake of the impatient reader,
let us give a preview on
how the cond-form works.

\index{Predicates!null}
\begin{figure}[!h]
	\begin{fmpage}{0.9\textwidth}
		\begin{verbatim}
		(define (avg xs)
  (let nxt [(s xs) (acc 0) (n 0)] 
    (cond [ (and (null? s) (= n 0)) 0]
	[ (null? s) (/ acc n) ]
	[else (nxt (cdr s) 
	      (+ (car s) acc) 
	      (+ n 1.0)) ] )))
		\end{verbatim}
	\end{fmpage}

	\begin{fmpage}{0.9\textwidth}
		\begin{verbatim}
		› scheme
		Chez Scheme Version 9.5.3
		Copyright 1984-2019 Cisco Systems, Inc.

		> (load "average.scm")
		> (avg '(3 4 5 6))
		4.5
		\end{verbatim}
	\end{fmpage}
	\caption{Going through a list}
	\label{going-through-a-list}
\end{figure}

\index{Conditional execution!cond}
The cond-form consists of a sequence
of clauses, where each clause has two
components: A condition and an expression.
The value of the cond-form will be given by
the first clause that has a true condition.
In listing~\ref{going-through-a-list}, the
clauses are:
\begin{enumerate}
	\item \verb| [ (and (null? s) (= n 0)) 0]| --
		The condition will be true when \verb|s| is the empty
		\verb|()| list and \verb|n=0|. This happens when the
		user types \verb|(avg '())|. In such a case, the
		cond-form returns 0.
	\item \verb|[ (null? s) (/ acc n) ]| -- This clause
		will be activated when the \verb|s| list is
		\verb|'()| empty, but \verb|n| is greater than 0.
		If \verb|n| were 0, the first clause would prevent
		the evaluator from reaching the second clause. The
		value of the second clause is \verb|(/ acc n)|,
		which is the average of the list.
	\item \verb|[else (nxt (cdr s) (+ (car s) acc) (+ n 1.0))]| --
		If the user executes the \verb|(avg '(4 5 6))| expression,
		\verb|nxt| will visit this clause with \verb|s= (4 5 6)|,
		\verb|s= (5 6)| and \verb|s= (6)|. Every time the \verb|nxt|
		falls into the third clause, \verb|(cdr s)| is executed, and
		\verb|s| looses an element, until it becomes empty and is
		captured by the second clause, which produces the answer.
\end{enumerate}

\section{Prolog Lists}\index{Lists!Prolog}
In his book about Lisp, Patrick Winston says: {\em A list is an
ordered sequence of elements, where ordered means that the
order matters.} In Prolog, a list is put between brackets,
and is supposed to have a head and a tail.\\

\begin{tabular}{p{5cm}p{3cm}p{3cm}}
	List &  Head & Tail\\
	\verb|[3,4,5,6,7]| & \verb|3| & \verb|[4,5,6,7]|\\ 
	\verb|[wo, ni, ta]| & \verb|wo| & \verb|[ni, ta]|\\
	\verb|[4]| & \verb|4| & \verb|[]|\\
	\verb|[3.4, 5.6, 2.3]| & \verb|3.4| &  \verb|[5.6, 2.3]|\\ 
\end{tabular}

\verb||\\
You can match a pattern of variables with a list. For instance,
if you match
\begin{quote}
	\verb/[X|Xs]/
\end{quote}
with the list \verb|[42,666,13]|, you will get \verb|X=42| and
\verb|Xs=[666,13]|, i.e., \verb|X| matches the head of the list,
and \verb|Xs| matches the tail. Of course, you can use other
variables instead of \verb|X| and \verb|Xs| in the pattern
\verb/[X|Xs]/. Thus, \verb/[A|B]/, \verb/[X|L]/, \verb/[First|Rest]/
and \verb/[P|Q]/ are equivalent patterns.\\
\index{Prolog!list pattern}\index{Prolog!list head/tail}
\verb||\\
\begin{tabular}{p{3cm}p{3cm}p{3cm}p{3cm}}
	Pattern & List &  X & Xs\\
	\verb/[X|Xs]/ & \verb|[42,666,13]| & \verb|42| & \verb|[666,13]|\\ 
	\verb/[X|Xs]/ & \verb|[666,13]| & \verb|666| & \verb|[13]|\\
	\verb/[X|Xs]/ &  \verb|[13]| & \verb|13| & \verb|[]|\\
	\verb/[X|Xs]/ &  \verb|[]| & no match &  no match\\ 
\end{tabular}

\verb||\\
Listing \ref{prolog:average} shows a Prolog program that calculates
the average of a list. A Prolog definition is a set of clauses, not
very different from the cond-form.

As you learned previously, when Prolog finds a solution through
one of the clauses, it offers the user another solution, obtained
by the use of the next clause in the definition. Computer scientists
say that Prolog backtracks to the point of choice, in order
to find multiple solutions. However, in the definition of
\verb|avg/4|, I know that if Prolog finds a solution through
the second clause, it will be useless to try the third
clause, since the empty \verb|[]| list will never unify
with the \verb/[X|Xs]/ pattern of the third clause. Therefore,
I cut\index{cut} the search tree by placing a exclamation mark
in the first and in the second clause. This is equivalent
to a declaration that \verb|avg/4| has only one solution.

Whenever Prolog finds the exclamation mark, it interrupt
the search for new solutions. The exclamation mark is
called {\em cut} by computer scientists. Since mathematical
functions have only one solution, the definition of a function
often requires a cut.\index{cut!exclamation mark}

\pagebreak
Let us test the predicate \verb|avg/4| given in
listing~\ref{prolog:average}. By the way, Prolog
is not very efficient in numerical computations.
Since we have Common Lisp available for number
crunching, there is no meaning in using Prolog
for arithmetic calculations.

\begin{verbatim}
› scheme
Chez Scheme Version 9.5.3
Copyright 1984-2019 Cisco Systems, Inc.

> (load "prolog.scm")
> (logic "-f prolog-src/average.pl")
CxProlog version 0.98.2 [development]

[main] ?- avg([3,4,5,6], 0,0, Average).
Average = 4.5

[main] ?- halt.
% CxProlog halted.
0
> (exit)
\end{verbatim}

\index{cut!average of a list}
\begin{figure}[!h]
	\begin{fmpage}{0.9\textwidth}
		\begin{verbatim}
		%% ?- consult('average.pl').

		avg([], Acc, 0, 0) :- !.
		avg([], Acc, N, A) :- N > 0, !, A is Acc/N.
		avg([X|Xs], Acc, N, A) :- S is Acc + X,
	  N1 is N + 1, avg(Xs, S, N1, A).
		\end{verbatim}
	\end{fmpage}
	\caption{Going through a list in Prolog}
	\label{prolog:average}
\end{figure}

\section{How to solve it in Prolog}
The title of this section is a homage to Helder
Coelho, the first author of the best book on
Prolog ever written. This book was printed by
the National Laboratory of Civil Engineering, where
Helder Coelho works in Portugal. 

Coelho's book is a collection of short problems,
proposed and solved by great programmers and computer
scientists. The whole thing is organized as a kind of
FAQ. The problems are interesting, and the solutions
are illuminating. The more important, the book provides
a lot of fun. 

\pagebreak
\paragraph{PROBLEM 1.} Verbal statement: Check whether H
is a member of a list L; search all members of a list L.

Prolog is the only language, where two lines code can be
interesting. The definition of \verb|memb| has two clauses.
The first clause says that \verb|H| is member of any list that
can be represented by the pattern \verb/[H|_]/, since it
is the first element of the list. The second clause says that
\verb|H| is a member of a list represented by the \verb/[_|T]/
pattern, if it is member of the tail T.\index{Prolog!member}
\begin{verbatim}
%% ?- consult('utilities.pl').

memb(H, [H|_]).
memb(H, [_|T]) :- memb(H, T).
\end{verbatim}\index{Prolog!\verb/[H:T]/ pattern}

As one would expect from the logic of the code,
Prolog answers yes when confronted with the
\verb|?-memb(2, [3,2,4])|, since 2 is member
of the \verb/[3,2,4]/ list. What is magic in
Prolog is that, when one submit the query
\verb|?-memb(X,[3,2,4])|, the inference engine
finds all members of the list.

The inference engine starts at the first clause,
and unify \verb|X| with the head of \verb|[3,2,4]|,
what produces \verb|X=3|. If the user does not like
this first solution, the second clause changes the
goal to \verb|memb(X,[2,4])| by eliminating the
refused answer. With the new goal, Prolog tries the
first clause again, which produces \verb|X=2|, the
second answer.
\begin{verbatim}
› scheme
Chez Scheme Version 9.5.3
Copyright 1984-2019 Cisco Systems, Inc.

> (load "prolog.scm")
> (logic "-f prolog-src/utilities.pl")
CxProlog version 0.98.2 [development]

[main] ?- memb(3, [2,3,4]).
yes? ;
no
?- memb(X, [2,3,4]).
X = 2 ? ;
X = 3 ? ;
X = 4 ? ;
no
?- halt.
% CxProlog halted.
0
> (exit)
\end{verbatim}

\paragraph{PROBLEM 7.} Specify the relation append (list
concatenation) between three lists, which holds if the
last one is the result of appending the first two.
\begin{quote}\index{Prolog!append}
	\begin{verbatim}
	%% (load "-f utilities.pl").

	applst([], L, L).
	applst([H|T], L, [H|U]) :- applst(T, L, U).
	\end{verbatim}
\end{quote}

This problem is an all time favorite. If you run the
goal \verb|?- applst(X,Y,[2,3,4,5])|, Prolog will find and
print all ways of cutting the list \verb|[2,3,4,5]|:
\begin{quote}
	\begin{verbatim}
	› scheme
	> (load "prolog.scm")
	> (logic "-f prolog-src/utilities.pl")
	CxProlog version 0.98.2 [development]

	[main] ?- applst(X, F, [2,3,4,5]).
	X = []
	F = [2,3,4,5] ? ;
	X = [2]
	F = [3,4,5] ? ;
	X = [2,3]
	F = [4,5] ? .
	...
	?- halt.
	% CxProlog halted.
	0
	> (exit)
	\end{verbatim}
\end{quote}

One can say that both Prolog and lisp use clauses to
process alternatives and options. The difference is
that, in Lisp, the choice is final. However, when
the choice reveals itself as inadequate, Prolog can
backtrack to the point of bifurcation and select
another path.\index{backtrack}

Consider the definition of applst, for instance. The
first alternative for the \verb|?- applst(X,F,[2,3,4,5])|
goal is the first clause, which makes \verb|X=[]|.
However, if the user is not happy with this solution,
Prolog backtracks to the point of choice, and selects
the second clause, where the \verb/[X|T]/ pattern can
accept lists of one or more elements, but not empty
lists. Let us suppose that the tail of the second
clause matches the first clause, i.e., \verb|applst(T,L,U)|
matches \verb|applst([],L,L)|. In this case, the query
will produce a second answer, \verb|X=[2]|. However,
the predicate \verb|applst(T,L,U)| can skip the first
clause, and match the second, which would make
\verb|T=[3]|, and \verb|X=[2,3]|, and so on.

\paragraph{PROBLEM 65.} [EMDEN, 1980] Write a program for
the game of Nim defined as follows.\index{Prolog!Game of Nim}

A position of the game of Nim can be visualized as a set of
heaps of matches. Two players, called {\em us} and {\em them},
alternate making a move. As soon as a player, whose turn is to
move, is unable to make a move, the game is over and that player
has lost; the other player is said to have won. A move consists
of taking at least one match from exactly one heap.

Ten years ago, I published a book called {\em Visual Prolog for
Tyros}, where I tackled this and a few other problems proposed
by Coelho. Here is my solution for the Nim problem:
\begin{quote}
	\begin{verbatim}
	%% (logic "-f prolog-src/nim.pl")

	% ap/3 -- the third argument is the concatenation
	%         of the first and the second
	ap([], L, L).
	ap([H|T], L, [H|U]) :- ap(T, L, U).

	% tk/3 -- takes a certain number of tokens from a heap
	tk([X],V,V).
	tk([X,X1|Y],V,[[X1|Y]|V]).
	tk([X|T],V,Y) :- tk(T,V,Y).

	mv(X,Y) :- ap(U,[X1|V],X),
       tk(X1,V,R),ap(U,R,Y).

       us(X,Y) :- mv(X,Y), \+ them(Y,Z).
       them(X,Y) :- mv(X,Y), \+ us(Y,Z).
	\end{verbatim}
\end{quote}
Coelho wrote: {\em A position of the game of Nim can be
visualized as a set of heaps of matches.} Let us represent
each heap as a list of integers:
\begin{quote}
	\verb|[1,1,1,1,1]|
\end{quote}
Therefore, the set of heaps will be a list of lists; e.g.
\begin{quote}
	\begin{verbatim}
	[  [1,1,1,1,1],
   [1,1],
   [1,1,1,1]]
	\end{verbatim}
\end{quote}

Two players, {\em us} and {\em them}, alternate making
moves. As soon as a player is unable to make a move,
the game is over, and that player has lost. A move consists
of taking at least one match from exactly one heap. In the
example, if you take three matches from the third heap,
you make a valid move for us, and the board becomes:
\begin{quote}
	\begin{verbatim}
	[  [1,1,1,1,1],
   [1,1],
   [1]]
	\end{verbatim}
\end{quote}
where you have removed three matches from the third heap.
To implement this project, we will use a programming
technique called {\em incremental development of systems}.
This technique is discussed in the first chapter of
{\em On Lisp}, a book by Paul Graham, that I strongly
recommend for your edification. 

First, you will implement and test a program that append
two lists. Since you are going to use this program both
to split and to concatenate lists, you need to test
both possibilities. 
\begin{quote}
	\begin{verbatim}
	› scheme
	Chez Scheme Version 9.5.3
	Copyright 1984-2019 Cisco Systems, Inc.

	> (load "prolog.scm")
	> (logic "-f prolog-src/nim.pl")
	CxProlog version 0.98.2 [development]

	[main] ?- ap([1,1,1], [2,2,2,2], L).
	L = [1,1,1,2,2,2,2]

	[main] ?- ap(X,Y, [1,1,2,2]).
	X = []
	Y = [1,1,2,2] ? ;
	X = [1]
	Y = [1,2,2] ? ;
	X = [1,1]
	Y = [2,2] ? .
	X = [1,1,2]
	Y = [2] ? ;
	X = [1,1,2,2]
	Y = [] ? .
	...
	?-
	\end{verbatim}
\end{quote}
The first query gives the result of concatenating two
lists. The lines that follow shows the many possibilities
of splitting a list. Them the first program seems to be
working properly. 

The next step is to write and test a program that takes
at least one match from a heap; of course, it can take
more than one match. After remove one or more matches
from the heap, \verb|tk/3| insert the modified heap into
a set of heaps. If you test the program, it will produce
the following solutions:
\begin{quote}
	\begin{verbatim}
	?- tk([1,1,1,1], [[1,1]], Resp).
	Resp = [[1,1,1],[1,1]] ? ;
	Resp = [[1,1],[1,1]] ? ;
	Resp = [[1],[1,1]] ? .
	...
	?- halt.
	% CxProlog halted.
	0
	\end{verbatim}
\end{quote}
N.B. The first clause of \verb|tk/3| makes sure that the
predicate will not insert an empty heap into the set of heaps.

I will leave the test of \verb|mv/2| to your discretion.
Below, you will find the result of a game between a human
player and the computer.

\begin{verbatim}
› scheme
Chez Scheme Version 9.5.3
Copyright 1984-2019 Cisco Systems, Inc.

> (load "prolog.scm")
> (logic "-f prolog-src/nim.pl")
CxProlog version 0.98.2 [development]

[main] ?- us([[1,1,1], [1,1]], Move).
Move = [[1,1],[1,1]] ? .
...
?- us([[1],[1,1]], Move).
Move = [[1],[1]] ? .
...
?- us([[1]], Move).
Move = [] ? .
...
?- exit.
0
> (exit)
\end{verbatim}

\chapter{Ninety-Nine Prolog Programs}
Werner Hett says in his article Ninety-Nine Prolog Programs:
\begin{quotation}
	The purpose of this problem collection is to give you the
	opportunity to practice your skills in logic programming.
	Your goal should be to find the most elegant solution of
	the given problems. Efficiency is important, but logical
	clarity is even more crucial. Some of the (easy) problems
	can be trivially solved using built-in predicates. However,
	in these cases, you learn more if you try to find your own
	solution.

	Every predicate that you write should begin with a comment
	that describes the predicate in a declarative statement.
	Do not describe procedurally, what the predicate does, but
	write down a logical statement which includes the arguments
	of the predicate. You should also indicate the intended data
	types of the arguments and the allowed flow patterns.

	The problems have different levels of difficulty. Those marked
	with a single asterisk (*) are easy. If you have successfully
	solved the preceeding problems you should be able to solve
	them within a few (say 15) minutes. Problems marked with two
	asterisks (**) are of intermediate difficulty. If you are a
	skilled Prolog programmer it shouldn't take you more than
	30-90 minutes to solve them. Problems marked with three
	asterisks (***) are more difficult. You may need more
	time (i.e. a few hours or more) to find a good solution. 
\end{quotation}\index{Prolog!Hett}

I did not followed Hett's advice to the letter. In order
to check for bugs in the compiler, I used cuts everywhere
to prevent backtracking. It is easy to restore the logic
purity of the programs: Just eliminate the cuts.
\index{Prolog!List examples}
\pagebreak
\begin{verbatim}
%% (load "-f p99.pl")

%% P01 (*): Find the last element of a list
%% ?- mylast(X, [1,2,3,4,5]).
%% X = 5

mylast(X, [X]) :- !.
mylast(X, [_|L]) :- mylast(X, L).

%% P02 (*): Find the last but one element of a list
%% ?- last_but_one(X, [1,2,3,4,5]).
%% X = 4

last_but_one(X, [X,_]) :- !.
last_but_one(X, [_,Y|Ys]) :- last_but_one(X, [Y|Ys]).

% P03 (*): Find the K'th element of a list.
% The first element in the list is number 1.
%% ?- element_at(X, [1,2,three,4,5], 3).
%% X = three
element_at(X,[X|_],1) :- !.
element_at(X,[_|L],K) :- K > 1,
   K1 is K - 1, element_at(X,L,K1).

   % P04 (*): Find the number of elements of a list.
   %% ?- length([1,2,3,4,5], L).
   %% L = 5

   length([],0) :- !.
   length([_|L],N) :- length(L,N1),
		  N is N1 + 1.

		  % P05 (*): Reverse a list.
		  %% ?- reverse([1,2,3,4],L).
		  %% L = [4,3,2,1]

		  reverse(L1,L2) :- my_rev(L1,L2,[]).

		  my_rev([],L2,L2) :- !.
		  my_rev([X|Xs],L2,Acc) :- my_rev(Xs,L2,[X|Acc]).

		  % P06 (*): Find out whether a list is a palindrome
		  %% ?- is_palindrome([s,u,b,i,d,u,r,a,a,r,u,d,i,b,u,s]).

		  %% Ignore spaces in Latin palindromes: Romans didn't use them:
		  %  subi dura a rudibus -- endure rudeness from peasants
		  %  roma tibi subito motibus ibit amor -- 
		  %    in Rome quickly with its bustle you will find love

		  is_palindrome(L) :- reverse(L,L).


		  % P07 (**): Flatten a nested list structure.
		  %% ?- flatten([3,4,[a,[b]], 5], L).
		  %% L = [3,4,a,b,5]

		  append([],L,L) :- !.
		  append([H|T], L, [H|U]) :- append(T,L,U).

		  flatten([],[]) :- !.
		  flatten([X|Xs],Zs) :- !, flatten(X,Y),
   flatten(Xs,Ys), append(Y,Ys,Zs).
   flatten(X, [X]).

   % P08 (**): Eliminate consecutive duplicates of list elements.
   %% ?- compress([2,3,3,3,4,4,5], L).
   %% L = [2,3,4,5]
   compress([],[]) :- !.
   compress([X],[X]) :- !.
   compress([X,Y|Xs],Zs) :-  X==Y, !, compress([X|Xs],Zs).
   compress([X,Y|Ys],[X|Zs]) :-  compress([Y|Ys],Zs).

   % P09 (**):  Pack consecutive duplicates of list elements.
   %% ?- pack([2,3,3,3,4,4,5], L).
   %% L = [[2],[3,3,3],[4,4],[5]]

   pack([],[]) :- !.
   pack([X|Xs],[Z|Zs]) :- transfer(X,Xs,Ys,Z), pack(Ys,Zs).

   transfer(X,[],[],[X]) :- !.
   transfer(Y,[X|Xs],Ys,[X|Zs]) :- X == Y, !, transfer(X,Xs,Ys,Zs).
   transfer(X,[Y|Ys],[Y|Ys],[X]).
\end{verbatim}

\begin{verbatim}
% P10 (*):  Run-length encoding of a list
%% ?- encode([a,a,a,b,b,b,b,b,c,d], Ans).
%% Ans = [[3,a],[5,b],[1,c],[1,d]]

encode(L1,L2) :- pack(L1,L),
	    transform(L,L2).

	    transform([],[]) :- !.
	    transform([[X|Xs]|Ys],[[N,X]|Zs]) :- length([X|Xs],N),
			     transform(Ys,Zs).

			     % P11 (*):  Modified run-length encoding
			     %% ?- encode_modified([1,1,2,2,2,2,2,3,3,4], G).
			     %% G = [[2,1],[5,2],[2,3],4]

			     encode_modified(L1,L2) :- encode(L1,L),
		      strip(L,L2).

		      strip([],[]) :- !.
		      strip([[Num,X]|Ys],[X|Zs]) :- Num == 1, !,
		       strip(Ys,Zs).
		       strip([[N,X]|Ys],[[N,X]|Zs]) :- N > 1,
			   strip(Ys,Zs).


			   % P12 (**): Decode a run-length compressed list.
			   %% ?- encode([a,a,a,b,b,b,b,b,c,d], Ans), decode(Ans, G).
			   %% Ans = [[3,a],[5,b],[1,c],[1,d]]
			   %%G = [a,a,a,b,b,b,b,b,c,d]

			   notlist(X) :- atom(X), !.
			   notlist(X) :- number(X).

			   decode([],[]) :- !.
			   decode([[Num,X]|Ys],[X|Zs]) :- Num == 1, !, decode(Ys,Zs).
			   decode([[N,X]|Ys],[X|Zs]) :- N > 1, !,
	   N1 is N - 1,
	   decode([[N1,X]|Ys],Zs).
	   decode([X|Ys],[X|Zs]) :- notlist(X),
			 decode(Ys,Zs).
\end{verbatim}
\index{Prolog!compression}
\pagebreak
\begin{verbatim}
% P13 (**): Run-length encoding of a list (direct solution) 
%% ?- encode_direct([1,1,2,2,2,2,2,3,3,4], G).
%% G = [[2,1],[5,2],[2,3],4]

encode_direct([],[]) :- !.
encode_direct([X|Xs],[Z|Zs]) :- count(X,Xs,Ys,1,Z),
		     encode_direct(Ys,Zs).

		     count(X,[],[],1,X) :- !.
		     count(X,[],[],N,[N,X]) :- N > 1, !.
		     count(X,[Y|Ys],[Y|Ys],1,X) :- X \= Y.
		     count(Y,[X|Xs],Ys,K,T) :- X == Y, !,
   K1 is K + 1, count(X,Xs,Ys,K1,T).
   count(X,[Y|Ys],[Y|Ys],N,[N,X]) :- N > 1.

   % P14 (*): Duplicate the elements of a list
   %% ?- dupli([1,2,2,3,3,3,5], L).
   %% L = [1,1,2,2,2,2,3,3,3,3,3,3,5,5]

   dupli([],[]) :- !.
   dupli([X|Xs],[X,X|Ys]) :- dupli(Xs,Ys).

   % P15 (**): Duplicate the elements of a list agiven number of times
   %% ?- dupli([1,2,2,3,3,3,5], 4,  L).
   %% L = [1,1,1,1,2,2,2,2,2,2,2,2,3,3,3,3,3,3,3,3,3,3,3,3,5,5,5,5]

   dupli(L1,N,L2) :- dupli(L1,N,L2,N).
   dupli([],_,[],_) :- !.
   dupli([_|Xs],N,Ys, K) :- K == 0, !, dupli(Xs,N,Ys,N).
   dupli([X|Xs],N,[X|Ys],K) :- K > 0, K1 is K - 1,
		dupli([X|Xs],N,Ys,K1).

		% P16 (**):  Drop every N'th element from a list
		%% ?- drop([1,2,3,4,5,6], 3, R).
		%% R = [1,2,4,5]

		drop(L1,N,L2) :- drop(L1,N,L2,N).
		drop([],_,[],_) :- !.
		drop([X|Xs],N,[X|Ys],K) :- K > 1, !,
       K1 is K - 1, drop(Xs,N,Ys,K1).
       drop([_|Xs],N,Ys, K) :-  drop(Xs,N,Ys,N).

       % P17 (*): Split a list into two parts
       %% ?- split([2,5,3,8,1], 3, X, Y).
       %% X = [2,5,3]
       %% Y = [8,1]

       split([X|Xs],N,[X|Ys],Zs) :- N > 0, !, N1 is N - 1,
		 split(Xs,N1,Ys,Zs).
		 split(L,0,[],L).
\end{verbatim}
\index{Prolog!slice}
\begin{verbatim}
% P18 (**):  Extract a slice from a list
%% ?- slice([1,2,3,4,5,6,7], 2,4, R).
%% R = [2,3,4]

slice([X|_],1,1,[X]) :- !.
slice([X|Xs],1,K,[X|Ys]) :- K > 1, !, 
   K1 is K - 1, slice(Xs,1,K1,Ys).
   slice([_|Xs],I,K,Ys) :- I > 1, 
   I1 is I - 1, K1 is K - 1, slice(Xs,I1,K1,Ys).

   % P19 (**): Rotate a list N places to the left 
   %% ?- rotate([1,2,3,4,5,6,7], 2, R).
   %% R = [3,4,5,6,7,1,2]

   rotate(L1,N,L2) :- N >= 0, !, 
   length(L1,NL1), N1 is N mod NL1, rotate_left(L1,N1,L2).
   rotate(L1,N,L2) :- 
   length(L1,NL1), N1 is NL1 + (N mod NL1), rotate_left(L1,N1,L2).

   rotate_left(L1,N,L2) :- N > 0, !,
    split(L1,N,S1,S2), append(S2,S1,L2).
    rotate_left(L,0,L).

    % P20 (*): Remove the K'th element from a list.
    %% ?- remove_at(X, [1,2,3,4,5,6], 3, L).
    %% X = 3
    %% L = [1,2,4,5,6]

    remove_at(X,[Y|Xs],K,[Y|Ys]) :- K > 1, !, 
   K1 is K - 1, remove_at(X,Xs,K1,Ys).
   remove_at(X,[X|Xs],1,Xs).


   % P21 (*): Insert an element at a given position into a list
   %% ?- insert_at(a,[1,2,3,4,5], 3, L).
   %% L = [1,2,a,3,4,5]

   insert_at(X,L,K,R) :- remove_at(X,R,K,L).


   % P22 (*):  Create a list containing all integers within a given range.
   %% ?- range(4,9,L).
   %% L = [4,5,6,7,8,9]

   range(I,K,[I]) :- I >= K, !.
   range(I,K,[I|L]) :- I < K, 
    I1 is I + 1, range(I1,K,L).


    % P23 (**): Extract a given number of randomly
    % selected elements from a list.
    %% (load "rnd.lisp") the Lisp function
    %% that generates random numbers.
    %% Study file "rnd.lisp" to learn how to
    %% implement new primitives.
    %% ?- rnd_select([1,2,3,4,5,6], 3, L).
    %% L = [6,2,4]

    rnd_select(_,K,[]) :- K < 1, !.
    rnd_select(Xs,N,[X|Zs]) :- N > 0,
    length(Xs,L),
    random(R, L), I is R + 1,
    remove_at(X,Xs,I,Ys),
    N1 is N - 1,
    rnd_select(Ys,N1,Zs).


    % P24 (*): Lotto: Draw N different random numbers from  1..M
    %% ?- lotto(6,49,L).
    %% L = [44,36,39,5,28,34]

    lotto(N,M,L) :- range(1,M,R), rnd_select(R,N,L).


    % P25 (*):  Generate a random permutation of the elements of a list
    %% ?- rnd_permu([1,2,3,4,5], L).
    %% L = [2,4,3,1,5]

    rnd_permu(L1,L2) :- length(L1,N), rnd_select(L1,N,L2).

\end{verbatim}
\index{Prolog!lotto}

\chapter{Shell}
\index{Shell}
Nia, a Greek young woman, has an
account on a well known social network.
She visits her friends' postings on a daily basis,
and when she finds an interesting picture or video,
she presses the {\em Like}-button. However, when
she needs to discuss her upcoming holidays on the Saba Island
with her Argentinian boyfriend, she uses the
live chat box. After all, hitting buttons and
icons offers only a very limited interaction tool,
and does not produce a highly detailed
level of information that permits the answering
of questions and making of statements.


Using a chat service needs to
be very easy and fun, otherwise all those teenage
friends of Nia's would be doing something else.
I am telling you this, because there are
two ways of commanding a computer.
The first is called Graphical User Interface (GUI)
and consists of moving the cursor with a
mouse or other pointing device and clicking
over a menu option or an icon,
such as the {\em Like} button.
As previously mentioned, a Graphical User Interface
often does not generate adequate information
for making a request to the computer. In addition,
finding the right object  to press
can become difficult in a labyrinth of
menu options.

The other method of interacting with
the computer is known as Shell and is similar to
a chat with one's own machine.

In your computer, there is a giant
program, called the operating system, which controls
all peripherals that the machine uses to
stay connected with the external world:
pointing devices,  keyboard,
video terminal, robots, cameras, solid
state drives, pen drives, and other peripherals.

In a Shell interface, Nia issues written instructions
that the operating system answers by fulfilling the
assigned tasks. The language that Nia
uses to chat with the operating
system is called {\em Bourne-again shell}, or
bash for short. This language has
commands to go through folders, browser
files, create new directories, copy
objects from one place to the other,
configure the machine, install
applications, change the permissions
of a file, create groups to organize
users and devices, etc.
When accessing the operating system through
a text-based terminal, a shell language is the main way
of executing programs and doing work on a computer.

The shell interface derives its name from
the fact that it acts like a shell
surrounding all other programs being run,
and controlling everything the machine performs.

To make a long story short,
it is much faster to complete tasks
using a shell than with graphical applications,
icons, menus and mouse. Another benefit of the
shell is that you can gain access to many
more commands and scripts than with
a Graphical User Interface.

In order not to scare off the feeble-minded,
many operating systems hide access
to the text terminal. In some distribution
of Linux, you need to maintain the \keys{Alt} key
down, then press the \keys{F2} key to open a
dialog box, where you must type the
name of the terminal you want to open.
If you are really lucky, you may find
the icon of the terminal  on the
toolbar.

\includegraphics[scale=0.8]{figs/terminal.png}

If the way of opening the text terminal
is not obvious, you should ask for help
from a student majoring in Computer Science.
You can teach her Russian, Samskrit, Javanese
or Ancient Greek, as a compensation
for the time that she will spend explaining
how to start a terminal in 
OS X or Linux. If you are afraid
of paying for a few minutes of tutorial
about the Bourne-again shell
with hundred hours of classes on
Russian, don't worry! The CS
major will give up after barely
starting the section on the alphabet. After all,
Philology is much more difficult
than Computer Engineering.
For details, read the tale ``The man who could speak
Javanese'' by Lima Barreto.

\paragraph{The prompt.}\index{Shell!prompt} 
The shell prompt
is where one types commands. The prompt
has different aspects, depending on the
configuration of the terminal.
In Nia's machine, it looks something like this:
\begin{quote}
	\verb|~$ _|
\end{quote}
Files are stored in folders. Typically, the
prompt shows the folder where the user is
currently working.

The main duty of the
operating system is to maintain the contents of
the mass storage devices in a tree structure
of files and folders.
Folders are also called {\em directories},
and like physical folders or cabinets,
they organize files.

A folder can be put inside another folder.
In a given machine, there is a folder 
reserved for duties carried out by the
administrator.
This special folder is called home
or personal directory.

Besides the administrator, a machine can
have other rightful users, each with
a personal folder. For instance, Nia's folder
on her Mac OS X has the \verb|/Users/nia|
path. You will learn a more formal
definition of path later on.


\paragraph{pwd \#}\index{Shell!pwd} 
The \verb|pwd| command informs
the user's position in the file tree.
A folder can be placed
inside another folder.
For example, in a Macintosh, Nia's home
folder is inside the \verb|/Users| directory.

One uses a path to identify
a nest of folders. In a path, a subfolder
is separated from the parent folder
by a slash bar. If one needs to know
the current folder, there is the \verb|pwd|
command. 
\begin{quote}
	\verb|~$ pwd         # shows the current folder. |\keys{Enter}\\
	\verb|/Users/nia|\\
\end{quote}
When Nia issues a command, she may
add comments to it, so her boyfriend that
does not know Bourne-again shell (bash)
can understand what
is going on and learn something in
the process. Comments are prefixed
with the \verb|#| hash char, as you
can see in the above chat. Therefore,
when the computer sees a \verb|#|
hash char, it ignores everything to
the end of the line.

\paragraph{mkdir wrk \#}\index{Shell!mkdir}
creates a \verb|wrk|
folder inside the current directory,
where \verb|wrk| can be replaced
with any other name.
For instance, if Nia is inside her
home directory, \verb|mkdir wrk| creates
a folder with the \verb|/Users/nia/wrk| path.

\paragraph{cd wrk \#}\index{Shell!change dir} 
One can use the
\verb|cd <folder name>| commands to enter
the named folder.
The \verb|cd ..|  command
takes Nia to the parent of the current
directory.
Thanks to the \verb|cd|
command, one can navigate through the
tree of folders and directories.

\paragraph{Tab.}\index{Shell!tab} If you want to go
to a given directory, type part of the directory
path, and then press \keys{Tab}.
The shell will complete the folder name for you.

\paragraph{Home directory.}\index{Home directory} One
can use a \verb|~| tilde to represent
the home directory. For instance,
\verb|cd ~| will send Nia to her
personal folder. The \verb|cd $HOME| has
the same effect, i.e., it places Nia inside
her personal directory.

\paragraph{echo \#}\index{Shell!echo} One can use the
\verb|echo| command to print something.
Therefore, \verb|echo $HOME| prints
the contents of the \verb|HOME|
environment variable on the terminal.

Environment variables store
the terminal configuration. For instance,
the \verb|HOME| variable contains
the user's personal directory identifier.
One needs to prefix the environment
variable with the \verb|$| char
to access its contents. Therefore,
\verb|echo $HOME| displays the
contents of the \verb|HOME| variable.

The instruction \verb|echo "Work Space" > readme.txt|
creates a \verb|readme.txt| text file and
writes the \verb|"Work Space"| string there.
If the \verb|readme.txt| file exists,
this command supersedes it.

The command \verb|echo "Shell practice" >> readme.txt|
appends a string to a text file. It does
not erase the previous content of
the \verb|readme.txt| file. Of course,
you should replace the string or the
file name, as necessity dictates.


Below you will find
an extended example of a chat between Nia and
the OS X operating system.
\begin{quote}
	\verb|~$ mkdir wrk   # creates a work space folder      | \keys{Enter}\\
	\verb|~$ cd wrk      # transfers action to the wrk file | \keys{Enter}\\
	\verb|~/wrk$ echo "* Work space" > readme.txt | \keys{Enter}\\
	\verb|~/wrk$ echo "This folder is used" >> readme.txt |\keys{Enter}\\
	\verb|~/wrk$ echo "to practice the bash" >> readme.txt | \keys{Enter}\\
	\verb|~/wrk$ echo "commands and queries." >> readme.txt | \keys{Enter}\\
	\verb|~/wrk$ ls      # list files in current folder. | \keys{Enter}\\
	\verb|readme.txt |\\
	\verb|~/wrk$ cat readme.txt  # shows the file contents.| \keys{Enter}
	\begin{verbatim}
	* Work space
	This folder is used
	to practice the bash
	commands and queries.
	\end{verbatim}
\end{quote}

\paragraph{ls \#}\index{Shell!ls} 
By convention, a file name has
two parts, the id and the extension. The id
is separated from the extension by a dot.
The \verb|ls | command lists all files
and subfolders present in the current folder.
The \verb|ls *.txt| prints only files
with the \verb|.txt| extension.

\paragraph{Wild card.}\index{Shell!wild card} The 
\verb|*.txt| pattern
is called wild card.
In a wild card, the \verb|*| asterix matches
any sequence of chars, while the \verb|?|
interrogation mark matches a single char.

\paragraph{ls -lia *.txt \#} prints detailed
information about the \verb|.txt| files,
like date of creation, size, etc.
\begin{quote}
	\verb|~/wrk$ ls -la |  \keys{Enter}
\end{quote}
\begin{verbatim}
total 8
drwxr-xr-x    3 edu500ac  staff    102 Sep 18 16:29 .
drwxr-xr-x+ 321 edu500ac  staff  10914 Sep 18 14:40 ..
-rw-r--r--    1 edu500ac  staff     74 Sep 18 16:30 readme.txt
\end{verbatim}

Files starting with a dot are called hidden files,
due to the fact that the \verb|ls| command does not
normally show them. All the same,
the \verb|ls -a| option includes the
hidden files in the listing.

\index{Permissions}
In the preceding examples, the first character
in each list entry is either a dash (-) or the letter d.
A dash (-) indicates that the file is a regular file.
The letter d indicates that the entry is a folder.
A special file type that might appear in a
\verb|ls -la| command is the symlink. It begins
with a lowercase l, and points to another location
in the file system.
Directly after the file classification comes the permissions,
represented by the following letters:
\begin{itemize}
	\item r -- read permission.
	\item w -- write permission.
	\item x -- execute permission.
\end{itemize}

\paragraph{cp readme.txt lire.fr \#}\index{Shell!cp} 
makes a copy
of a file. You can copy a whole directory
with the \verb|-rf| options, as shown below.
\begin{quote}
	\begin{verbatim}
	~/wrk$ ls
	readme.txt
	~/wrk$ cp readme.txt lire.fr
	~/wrk$ ls
	lire.fr  readme.txt
	~/wrk$ cd ..
	~$ cp -rf wrk wsp
	~$ cd wsp/
	~/wsp$ ls
	lire.fr  readme.txt
	\end{verbatim}
\end{quote}

\paragraph{rm lire.txt \#}\index{Shell!rm} 
removes a file.
The \verb|-rf| option 
removes a whole folder.
\begin{quote}
	\begin{verbatim}
	~/wsp$ ls
	lire.fr readme.txt
	~/wsp$ rm lire.fr
	~/wsp$ ls
	readme.txt
	~/wsp$ cd ..
	~$ ls wrk
	lire.fr readme.txt
	~$ rm -rf wrk
	~$ ls wrk
	ls: wrk: No such file or directory
	\end{verbatim}
\end{quote}

\paragraph{mv wsp wrp \#}\index{Shell!mv} changes
the name of a file or folder, or
even permits the moving of a file or
folder to another location.
\begin{quote}
	\begin{verbatim}
	~$ mv wsp wrk
	~$ cd wrk
	~/wrk$ ls
	readme.txt
	~/wrk$ mv readme.txt read.me
	~/wrk$ ls
	read.me
	~/wrk$ cp read.me wrk-readme.txt
	~/wrk$ ls
	read.me wrk-readme.txt
	~/wrk$ mv wrk-readme.txt ~/Documents/
	~/wrk$ ls
	read.me
	~/wrk$ cat ~/Documents/wrk-readme.txt
	Work space
	This folder is used
	to practice the bash
	commands and queries.
	\end{verbatim}
\end{quote}

\paragraph{Copy to pen drive.}\index{Pen drive}
In most Linux distributions, the pen drive is
seen as a folder inside the \verb|/media/nia/|
directory, where you should replace \verb|nia|
with your user name. In the Macintosh, the
pen drive appears at the \verb|/Volume/| folder.
The commands {\bf cp}, {\bf rm} and {\bf ls}
see the pen drive as a normal folder.


\section{Package managers}\index{Package managers}
A package
manager is a tool that works with
core libraries to handle, even though poorly,
the installation and removal of
software. Each operating system has its own
package manager. Even if you stick to a single
operating system, you will have to
deal with many package managers
and different philosophies concerning
software installation.


I will teach you how to use three
installation tools: apt-get,
homebrew and git clone. However,
this tutorial will not
completely cover the many subtleties
involved, in order to make your machine usable.
You will really need the help of
that girl who is majoring in
Computer Engineering.
Tell her that, if she teaches you
how to configure apt-get install,
you will teach her the ancient
Javanese script.


\section{github}\index{wamcompiler}
In the jargon of tech-oriented people,
a person who uses computers is called
{\em user}, but engineers
prefer the term {\em looser}, as
the latter is an affectionate way
of calling those individuals who need to work
with a computer, but in fact have no skills. By the way,
looser is a loser who can't spell ``loser''.
Fortunately for us {\em losers}, there exists
a repository called github. Let us see how
to download the source code for this document
from github:
\begin{quote}
	\verb|https://github.com/FemtoEmacs/lisp-plus-prolog/|
\end{quote}

\section{sudo}

Installation often requires that the
system copy files to the \verb|/usr/local/|
folder. However, you must give special
permission for this operation to be performed,
otherwise the possibility remains open for
a virus to make similar copies in critical folders.
The \verb|sudo| command informs the machine that you
are a superuser. To avoid being tricked by a virus
or an unauthorized user, the machine asks for the
password, and if you type it correctly,
the system performs the installation.

Let us see an example of sudo in action.
If you decide to use Chez Scheme, instead
of Racket, you may need to install a front end
library called \verb|libncurses|. Let us understand
this issue. When using Chez Scheme, you need to
edit the command line. The libncurses library provides
the functionality necessary for such a task.

There is a large choice of front-ends. Some front-ends 
work with a raster graphics image, which  is a dot matrix 
data structure that represents a grid of pixels. 
There are also front-ends that are specialized in
showing letters and other characters via an appropriate
display media. The latter sort of front-end is called
text-based user interfaces, while the former is called
graphical user interface, or GUI for short. Racket offers
both user interfaces, while Chez Scheme offers only
a text user interface, which requires \verb|libncurses|.

Text-based user interfaces are more comfortable on the
eyes, since they  provide sharper and crisp alphabetical
letters. On the other hand, GUI allows for font
customization. At present, Chez Scheme offers only a text
user interface based on ncurses. Therefore, search the web
for the distribution site and download the most recent
version of this front-end.\\

\includegraphics[scale=0.6]{figs-prefix/download.jpg}

Unpack and make the distribution archive as shown below:
\begin{quote}
	\begin{verbatim}
	~$ tar xfvz ncurses-6.0.tar.gz
	~$ cd ncurses-6.0/
	~/ncurses-6.0$ ./configure --prefix=/usr/local \
			   --enable-widec
			   ~/ncurses-6.0$ make
			   ~/ncurses-6.0$ sudo make install
	\end{verbatim}
\end{quote}
The \verb|ncurses-6.0| directory contains
many files in the C  computer language.
In order to become usable, these files
must be compiled into a set of libraries
containing instructions written in a
language that the computer understands.
You do not need any knowledge of C to
perform the translation to machine
language. All instructions in the
batch job for performing the compilation
are written in  \verb|Makefile|. However,
\verb|Makefile| on its on is not
sufficient. For the operation to be
successful, the system needs also to know which
computer you are using. The \verb|configure|
contains the program that analyzes your
hardware.

The \verb|--enable-widec| option customizes
the text terminal for chars with
diacritics, so you can write words
like {\em café} and {\em façade}.

The installation requires that the system copy files
to the \verb|/usr/local/| folder, and the option
\verb|--prefix=/usr/local| conveys this information.

\chapter{The man who knew Javanese}
\index{Javanese}
In a doughnut shop the other day,
I was telling my friend Castro
how I tricked respectable members of the
community not only to make a living,
but also rise to a higher social position.

For instance, while I was living in Manaus,
I had to hide my professional degree to win
over the clients' trust. People, who never would
come to a physician's practice or lawyer firm,
flocked to my wizard and fortune teller's office.
So, I was telling him this story.

My friend listened to me in silence,
enraptured by my words,
appreciating the account of picaresque adventures,
that seemed as though they had been
extracted from the Story of Gil Blas.
During a conversational pause,
after both drying our glasses,
he remarked randomly:

``You've been living an extremely funny life, Castelo!''

``It's the only kind of life worth living\ldots
I could not immagine
myself having a single occupation.
It is extremely boring
to go to work in the morning,
come back in the evening, read the paper
on the sitting room armchair,
and go to bed at eleven fifteen.
Don't you think?
In fact, I can't imagine how I have been
able to stand my job at the Consulate!''


``Yes, the existence in the civil service
must be tedious. But that's not
what amazes me. I wonder how have you been
able to pass through so many adventures here,
in this monotonous and bureaucratic Brazil.''

``You will be surprized, my dear Castro,
to learn that even here, in this country,
one can write quite interesting pages on a
memoir. Believe it or not,
I was even a teacher of Javanese!''

``When was that? Here, after you retired
from your job at the Consulate?''


``No, it was a long time ago.
What's more, I got my job at the
Consulate as a consequence of my
teaching duties.''

``That story I want to hear! In the mean time,
let us have a few more drinks?''

``Yes, I accept another glass of beer.''

We sent for another bottle, filled our glasses
and I continued:

``I had just arrived in Rio and I was literally broke.
My life was consumed fleeing
from one boarding house to
the other on rent day, without
idea of how to earn a living.

Of course, I didn't have money
to pay for breakfast.
Even so, I was in a café, but only to
read through the job openings section
of the newspaper left on a table
for the customers' convenience. Then, I came across
the following advertisement:


``Needed: A JAVANESE TEACHER.  Interviews at, etc.''

So, I told myself: ``This certainly is a position
where you won't find many applicants. If only I
could comprehend a few words of Javanese,
I'd apply.''

I left the coffee shop in a waking dream,
where imagined myself in a brave new world,
as a Javanese teacher,
earning more than just a decent living,
driving my car,
without unhappy meetings with debt collectors.

When I awoke from my daydream,
I found myself in front of the public library.
I went in and gave my cap to a
person that looked like an attendant,
since I knew about the custom of the
upper class to leave their hats at
the hat stand. I did not have
a hat, so I left my cap. Of course,
I never saw my cap again.

I didn't really know what
book I was going to ask for.
In any case, this was irrelevant,
because my name was nationally blacklisted,
since I never returned the books to
the highschool library. Nevertheless,
I climbed the stairs to browse
an Encyclopedia,
volume J and look up the entry on
Java and the Javanese dialect. 
In a few minutes, I knew that Java was a
big island of the Greater Sunda Islands,
at the time a Dutch colony.

%%\index{Brahmic script}
I also learned that Javanese is an agglutinative
language of the Malayo-Polynesian family.\index{Polynesian languages}
It has a noteworthy literature written in
characters derived from the old Hindu alphabet.
The Javanese alphabet is a modern variant
of the Kawi script, a Brahmic script developed
in Java around the ninth century.
In the past, it was widely used in religious literature,
which was written on palm-leaves.
This kind of manuscript came to be known
as  lontar.
The Encyclopedia gave a list of books for further
reading and I had no doubt of what I needed
to do, and consulted one of them.
I made a copy of the alphabet and its phonetic
transcription and left. I wandered the streets
walking aimlessly chewing over the letters as
they were gum.

Hieroglyphs danced in my head. From time to time,
I would look at my notes. Then I would
go into Tijuca park, and wrote
those strange looking characters
with a stick in the sand to keep
them vividly in my memory
and accustom my hands in writing them.

I would enter my building late at night in
an attempt to avoid coming across the
landlord, who could ask for the rent.
Even in my bedroom I kept chewing over
my Malayan A-B-C. My determination was such
that I knew it all by heart with the sunrise.
I convinced myself that Malay was the
easiest language in the world, and that Javanese
was as close to Malay as Portuguese to
Spanish, which simply is not true.
The reality is that Malay is not so
easy, and it is  as
far from Javanese as English from German.

I tried to leave home as early in the morning
as possible to avoid my landlord.
But to my dismay, it was not early enough,
for I couldn't escape meeting the man
in charge of asking for the room rent.

``Mr. Castelo, when are you going to
pay your rent bill?''

I aswered him with the most heart filled hope:

``Soon\ldots Wait just a little longer\ldots
Be patient\ldots I will be appointed as a
Javanese teacher and\ldots''
At that moment, the man cut in into my stuttering
excuses.

``What in hell is that, Mr. Castelo?''

I enjoyed the amusement
and proceeded to invest
in the man's patriotism:

``It's a language from the distant reaches of Timor.
Do you know where it is?''

Oh, simple minded fellow! The man forgot all about
my overdue bill and told me in a strong accent
from Portugal:

``I am not sure, Mr. Castelo. If I remember rightly,
Portugal has or had a colony with that name
close to Macau. But do you
really know such a thing, Mr. Castelo?''

Incentivated by this initial result from
my Javanese studies,
I started to fulfil the requisite of the advertisement.
I decided to offer my services as a teacher of
that transoceanic idiom. I composed the letter of
acceptance, then I went to the newspaper
office to left the document there.
With that, I returned
to the library to resume my Javanese studies.
I didn't make any progress that day. In fact, I
felt that the alphabet was the only knowledge
required from a Javanese teacher. Perhaps,
I was overly indulged with the
bibliography and history of the
language that I was going to teach.

Two days later I received an answer to my correspondence,
in which was stated
that I should go to the residence of
a certain Dr. Manuel Feliciano Soares Albernaz,
Baron of Jacuecanga,
Conde de Bonfim street.
I can't recall the house
number. Remember that  I was extremely focused
on Malayan studies to register the details
for history, as for example the house number.
But don't be fooled into thinking that
my lack of memory for particularities,
such as house numbers or names of historical
figures, renders the story as a figment
of my immagination. More than one scholar
told me that the  name of a certain
Prince Kalunga \index{Prince Kalunga}
is Aji Saka. \index{Aji Saka}
There you are.

Besides the alphabet, I knew the names
of a few authors. I also could say, ``How are
you?'', and knew a few grammar rules.
All this knowledge was supported
on around twenty words.

You can't imagine how hard I tried
to get the money for the bus fare! Javanese
is easier than loaning money,
you can be sure. Failing to raise
the necessary funds for the trip from
one neighborhood to the other, I walked.
Of course, I arrived in a state of dripping sweat.
With a motherly
affection, the old
mango trees standing in front
of the baron's house received, harbored
and cooled me down with the fresh air of shade.
It was the first time in all my life
that I felt sympathy for Nature.

The house was huge, but ill kept.
However, it was that way more from
despondency and weariness of living
than from its owner's poverty.
The walls hadn't been
painted for years, and what was left
of their coats was peeling away.
The eaves around the edges of the
roof were made up of old fashioned tiles,
some of which were missing, creating
as effect of decaying false teeth.
In the garden, the
flatsedge and other weeds had expelled the 
begonias. However, the more resistent crotons
continued to bloom with their purplish leaves.

I knocked. It was awhile before 
an aged African Negro came to the door. His white
beard and hair gave him an appearance of
fatigue and suffering.

In the parlor there was a row of gold framed portraits,
which showed bearded gentlemen and profiles of
sweet ladies holding huge fans and dresses
that looked like balloons ready to ascend in the air.
From among the many items,
to which the dust gave more antiquity and respect,
what impressed me most was a big, beautiful,
porcelain vase from China or India, I never
learned to differentiate their origins.
Anyway, the purity of the material,
its fragility, the naiveté of its contour
and its dim lustre that appeard to be
as the moonlight suggested to me that
the artwork was fashioned by the hands
of a dreaming child for
the enchantment of old men's  disillusioned eyes.

As I described above, I was inspecting the furniture,
while waiting for the master of the house.
He delayed quite a long time. Then, he came.
Full of respect, I watched the elderly man
approach haltingly,
a large handkerchief in his
hand, inhaling an old fashioned peppermint
essence venerably.

Although I wasn't sure he was my pupil to be, I felt
that it would have been wicked to deceive the
aged gentleman, whose ancient aspect made
me think of something august, sacred.
I hesitated for a moment. Should I invent
an excuse and leave? But finally
decided to wait and see.

``I am,'' I said to break the ice,
``The Javanese teacher you asked for.''

``Sit down,'' answered the old gentleman.
``Are you from Rio?''

``No, Sir. I'm from Canavieiras.''

``What?'' He said. ``Speak louder, my hearing is impaired.''

``I am from Canavieiras,'' I repeated putting emphasis
upon each word.

``Where did you go to school?''

``In the city of Salvador, in the great state of Bahia.''

``And where did you learn Javanese?'' he
asked with that dodgedness so common among the elderly.

I was not expecting for such a question,
but I made up a lie about a
Javanese father, who had come to Bahia
on a freighter. Liking what he saw,
my phony father decided to
settle there in Canavieiras as a fisherman.
He married, prospered and taught me Javanese.

``Did he believe you?'' asked my friend
who up to that moment had remained silent.
``And your features?''

``I'm not very different from a Javanese''
I contested. ``With my thick, straight, black hair
and tanned skin, I could very well pass for
a Indonesian halfbreed. You know very well
that among us there are all kinds of nationalities
-- Indians, Malayans, Tahitians, Madagascans,
Guanches\footnote{Guanches were the Berber aboriginal
inhabitants of the Canary Islands.}
and even Goths. Our people are an amalgamation
of races and types to make the whole world envy us.''

``O.K..'' My friend agreed. ``Please, go on.''

``The old gentleman,''  I resumed,
``listened intently and examined my physique
for a long while. Then he concluded that I was
indeed the son of a Malayan, and asked me softly:

``Well, do you really want to teach me Javanese?''

The answer came unwittingly: ``Yes.''

``You must be astounded,'' added the Baron
of Jacuecanga, ``that I, at my age, should
still have a wont for learning.''

``I'm not surprized at all. There have been
noteworthy examples of late learners,
as Socrates, who started flute at seventy.''

``What I really want, Mr.\ldots?''

``Castelo,'' I supplied.

``What I really want, Mr. Castelo,
is to fulfill a family pledge.
I don't know whether you realize that I'm the
grandson of State Secretary Albernaz, the same
man who accompanied Peter,
the first emperor of Brazil,
into exile, after his abdication.
When he came back from London,
the Secretary brought with him a book written in a strange
language, onto which he bestowed great value.

The old volume had been given to my grandfather
by an Indian or a Siamese sailor in
return for what received favor I do not know.

Before he died, my grandfather called
my father to his deathbed and told him
the story that follows.

``Son, do you see this book here, written in Javanese?
Well, the person who
gave it to me believed that it would bring its owner
happiness and deliverance from evil. I don't know
whether you should believe in such a legend or not.
In any case, keep the book, and if
the good omens prophesized by the oriental wiseman
come true, teach your son to read it,
so that our family line should prosper.''

``My father,'' proceeded the old baron, ``didn't take
the story to heart. Nevertheless, he maintained
the book in safe keeping.
When he lay sick and suspected that the end of his
life was near\footnote{Before the Scholar among
my readers accuse me of plagiarism,
I avow that I am fluent in Ancient Greek,
and the first book I read in my life
was the Anabasis.},
he gave it to me repeating the
dying words of his own father.''

In the beginning, I 
paid little attention to the book. I threw it 
in a corner, and got on with life.
I even forgot it existed. Lately,
so many misfortunes have befallen me,
that I remembered the old family heirloom.
I must read and understand the book, if I want to
preserve my last days from witnessing the total
ruin of my posterity. Of course, to read the book,
I need to master the Javanese language.
There you have all of it.''

He became silent. I noticed his eyes glistened
with tears.
He wiped them discreetly and asked me if I
would care to see the book. I gave a yes.
He summoned the maid, passed her the instructions
to find and fetch the book,
while he told me how he had lost all his children,
nephews and nieces. Only one married daughter
remained alive. From her numerous offspring,
only one son survived, a  weak and infirm boy.

The book arrived, an old,
large volume, presented in
quarto bound in leather and printed in huge
letters on yellowed paper.

Since the title page
was missing, I was unable to determine
the date of publication. Fortunately,
the preface was written
in English. There I read that the book
contained the stories of a Prince Kalunga,
a renowened writer.

I explained this entirety to the baron.
Ignorant of the fact that I arrived at 
this deliverance of knowledge
from the English preface, he was
very impressed with my erudition on
Javanese culture. I browsed the pages
with the look of someone familiar
with that language that most people
would not be able to tell apart from
Sanskrit or Hindi.

Before parting company, the baron and I
agreed upon my fees
and the class schedules.
To fulfill my part in the contract,
I should teach my old pupil to read
the book in a year.

A few days later, I gave the gentleman
the first private lesson, but the old man
was not diligent, not even at my level
of interest. At the
end of the class, he could not distinguish
one character of the abugida from the other,
or trace the five first letters of the
hanacaraka sequence.

Traditionally, the Javanese syllabary
is taught through a poem of 4 verses
narrating the myth of Aji Saka.\index{Javanese alphabet}
\begin{quote}
	\includegraphics[scale=0.3]{figs/ha.png}
	\includegraphics[scale=0.3]{figs/na.png}
	\includegraphics[scale=0.3]{figs/ca.png}
	\includegraphics[scale=0.3]{figs/ra.png}
	\includegraphics[scale=0.3]{figs/ka.png} --
	There were two messengers
\end{quote}
\begin{quote}
	\includegraphics[scale=0.3]{figs/da.png}
	\includegraphics[scale=0.3]{figs/ta.png}
	\includegraphics[scale=0.3]{figs/sa.png}
	\includegraphics[scale=0.3]{figs/wa.png}
	\includegraphics[scale=0.3]{figs/la.png} --
	with mutual hatred.
\end{quote}
\begin{quote}
	\includegraphics[scale=0.3]{figs/pa.png}
	\includegraphics[scale=0.3]{figs/dha.png}
	\includegraphics[scale=0.3]{figs/ja.png}
	\includegraphics[scale=0.3]{figs/ya.png}
	\includegraphics[scale=0.3]{figs/nya.png} --
	One was as mighty as the other.
\end{quote}
\begin{quote}
	\includegraphics[scale=0.3]{figs/ma.png}
	\includegraphics[scale=0.3]{figs/ga.png}
	\includegraphics[scale=0.3]{figs/ba.png}
	\includegraphics[scale=0.3]{figs/tha.png}
	\includegraphics[scale=0.3]{figs/nga.png} --
	Here are the corpses.
\end{quote}


In the poem, each syllable is written
with a different letter. After learning
the first verse, the student has learned
five letters.  Five more letters are
presented in each sebsequent verse. 

It took a whole month for the Baron of
Jacuecanga to learn the first and second
verse. What is worse, only one day was
enough for him to forget everything.
So, both teacher and student started a long
cycle of learning and forgetting.

I think that the Baron's daughter and her husband
didn't know anything about the book until they
become curious of my activities in their home.
When I explained the matter to them, they took it
lightly. They laughed about the behavior of their
elderly relative, said that he probably had dementia,
and that learning languages could slow the progress of
the disease. 

You won't believe it, my dear Castro, but the
son-in-law came to admire and respect the teacher
of Javanese. He kept saying: ``It is amazing how
my father-in-law's tutor could grasp a culture
so different from our own! Yet, he is
so young. If I were as knowledgeable as he,
I would be a scholar at Cambridge or at the Sorbonne.

The husband of Mrs. Maria da Gloria,
that was the name of the Baron's daughter,
was a judge, a powerful and well connected man.
Even so, he never tired of showing
his admiration for my knowledge of Javanese.

The Baron also seemed happy with me.
But after two months he gave up learning
the language. Instead of reading the book
himself, he concluded that it would be
enough to understand its contents
in order to fulfill the pledge he made to his
dying father. Certainly, the powerful
spiritual forces behind the book would
not be opposed to the humble request for
the services of a translator.
He would hear my rendition of the story of
Aji Saka, avoiding the effort that his
old brain was not in shape to bear.

You are certainly aware that even today
I don't know Javanese. But as every
learned man in this country, I had heard
the story of King Aji Saka, or Prince Kalunga,
as Brazilians prefer to call the hero.
In my rural village, at night,
in my room illuminated by candles and
kerosene lamps, my mother used to read
a translation of the Indonesian
legend into Latin until I would fall asleep.
Therefore, it was not very difficult to
patch together memories from my childhood,
fill the gaps, and present the result
to the old man as coming directly from the book.
He would hear those myths in ecstasy, as
though an angel were providing the rendition.
And my reputation
increased among the members of the Baron's
family and its circle of acquaintainces.
The Baron raised my salary, and I started
living an easy life.

An unexpected fact contributed to my
prestige. The Baron received an inheritance
from a Portuguese relative that he  didn't
even know existed. Of course, he assigned the
happy event to the Javanese book.
I myself almost believed that this was so.

As time went by, I stopped myself from
feeling remorse and guilty due to my deceiving
the naive man and his family. In any case,
I was terrified at the prospect of coming
across some horrible person that could
speak the Javanese dialect. My terror
increased when the Baron wrote a letter
to the Viscount of Caruru suggesting my
name for an international affairs career.
I advanced every kind of objection to the
idea. ``I am very ugly, uncouth and gaunt.
My knowledge of French
is faulty. I am not elegant.
I don't know the protocols.''

He insisted: ``Physical aspect doesn't matter, young man.
Go ahead! Everybody speaks French, German and English.
But only you know Javanese.''

So, I went to the interview with the Viscount,
who sent me to the Ministry of Foreign Affairs
with more than one recommendation letter.
After all, everybody wanted to have the
honor of giving me a recommendation letter.
I was a huge success when I arrived at
the office of the director
of fucking nothing\footnote{A literal translation
of the Portuguse acronym for ASPONE -- Assessor de
Porra Nenhuma.}.

The director called all the department
heads: ``Look at this! The man
knows Javaness.  What a prodigy!''

The heads of various departments introduced me
to clerks and amanuensis.
One of these lesser officials of the Ministry
gave me a look with more envy and hatred than
admiration. But everybody else kept saying:
``Do you really know Javanese? Is it difficult?
I believe that nobody else speaks that language here.''

The clerk who had looked at me with undeserved
hatred approached the group of admirers and interrupted
the applauses with a cold comment: ``That is true,
nobody knows Malay or Indonesian here. But I speak
Kanak, a language of the New Caledonia. Do you
know Kanak?

I told him I didn't and proceeded
to the Minister's office.

The high authority got up, adjusted the
glasses on his nose, and asked point blank:
``So, do you speak Javanese?''
My answer was a loud {\em yes}.
He wanted to know where I learned the
language. I told him about my Javanese father.
The Minister was sincere and direct.
``You can't go into the Diplomatic Service.
Brazilian Diplomats must be blond with
blue eyes. Of course, there is the affirmative
action for blacks and indians. But although
your skin is dark, you cannot apply,
since you are Javanese, not Indian.
We could send you to a consulate in
Asia or Oceania, but I am afraid
that there is no opening now.
However, if a position appears, you will
get it. Meanwhile, you will be attached directly
to my Ministry. By the way,
I want you to go 
to Basel sometime next year, in order
to represent our country in a congress
of Linguistics. Read the books of
Hove-Iacque, Max Müller, and other
good authors in the field.''

Can you follow me? I didn't know Javanese
and could barely ask for lunch in French
or English, but have a good job, and would
represent my country in a meeting of scholars!

The old Baron came to pass,
and the book went to his son-in-law,
with the intention that it be given
to his grandson when the boy come to age.
The deceased Baron also left me a sum
in his will.

I cannot say that I didn't try to learn
tha Malayo-Polynesian languages, but to
no avail. The effort of learning those
languages is way out of my reach.
Besides this, I had more pressing
business to take care of: Eating well, dressing
elegantly, and reading Comics.

Anyway, even if you don't read much,
it is advisable to have shelves of books
and many volumes of journals in your
office. Therefore, I subscribed to
the Revue Anthropologique et Linguistique,
the Proceeding of the English Oceanic
Association, and the Archivo Glottologico
Italiano. Of course, subscribing to those
scholarly journals without reading
them did add anything to my learning
of languages.


Yet, my reputation did not stop increasing.
People who I met on the street would greet me
with the comment:

``There goes the man who knows Javanese.''

In book stores, grammarians often
would consult me about the position
of pronouns in the dialect spoken
on the island of Sonda.
Scholars sent me letters  from all
over the world and newspapers would write 
articles about me. On an occasion,
I had to refuse a
group of well to do young students
who wanted to learn Javanese at any cost.

Do you remember that commerce paper where
I found the advertisement? The editor
asked me to  write an
article on the classical
and modern Javanese literature.
I obliged.

``How could you, since you avow to be ignorant
of the language, let alone the literature?''
asked  Castro.

``That was not very difficult. I started
with a detailed description of the island
of Java. Dictionaries and
geography textbooks provided me with the necessary
information.
Then, I quoted American, German and French
authors.''

``Did anybody ever put your knowledge in question?'' My friend asked.

``Never, although I was almost caught out on
a particular occasion. The police arrested
an individual, a sailor, a brown fellow who
spoke a strange language. They called a number
of different interpreters, but to no avail.
Finally the police got in touch with me,
the Javanese scholar, with all the respect
my position deserved. I delayed in attending
the request, but in the end I could no longer
escape the compromise. Fortunately, the man
had been released, thanks to the intervention
by the Dutch consul to whom the sailor
could explain his case through the use
of half a dozen or so Dutch words.
The sailor was indeed Javanese.''

The moment finally arrived for my
attending the congress. So, there
I was on my way to Europe.
It was marvelous! I attended to
the opening ceremony and the invited
paper presentations. The organizers
registered me in the Tupi-Guarani
section. After a couple of days at
going from presentation to presentation,
I concluded that the whole thing was
not for me, and left for Paris.
Before taking the train, however,
I was careful enough to convince a
journalist to publish my portrait
and a short article about me. Thus,
I had a way to prove that I was in
Basel, not in Paris.

Eventually, I returned to Basel for
the closing ceremony. The organizers
of the congress assumed that my
abscence was due to their mistake
of assigning me to the Guarani section.
I accepted the apology, but I still
was not able to write that treatise
on Javanese that I promised them.

After the congress, I paid for the publication
of German and Italian translations of the article
from the Basel paper. The readers of these
translations offered a banquet in my honor.
The banquet was presided by Senator Gorot.
My contribution cost me all the money
the Baron left me in his will.

The money was well spent. After all,
I became a celebrity, and after six
months I was appointed Consul in
Havana, where I lived for many years,
in order to perfect my proficiency
of the Polinesian languages.

``It is fantastic,'' observed Castro, holding
tightly to his glass of beer.

``Look! Do you know what I would like to
be right now?''

``What?''

``A famous genetic or computer engineer. Let us
go for it?''

``I am in.''


\chapter{Arithmetic operations}
Scheme has many resources for writing and documenting programs.
However, this book follows Occam's advice and use only those
tools that are absolutely necessary. In this chapter, you will
shadow Nia, while she determines which is the minimum set of
tools she needs to perform numerical calculations.

Nia entered the editor and pressed \verb|C-x C-f| 
to create the \verb|celsius.scm| buffer. 
She typed the  program below into the buffer. 
\begin{verbatim}
; File: celsius.scm
; Comments are prefixed with semicolon
;; Some people use two semicolons to
;; make comments more visible.

(define (c2f x)
   (- (/ (* (+ x 40) 9) 5.0) 40)
   );;end define

   (define (f2c x)
   (- (/ (* (+ x 40) 5.0) 9) 40)
   );;end define
\end{verbatim}
Function c2f converts Celsius readings to Fahrenheit.
The \verb|define| macro, which defines a function, 
has four components, the function id, a parameter list,
an optional documentation, and a body that contains
expressions and commands to carry out the desired
calculation. Comments are prefixed with a semicolon.
In the case of \verb|c2f|, the id is \verb|c2f|,
the parameter list is \verb|(x)|, and the body 
is \verb|(- (/ (* (+ x 40) 9) 5.0) 40)|.

Lisp programmers prefer the prefix notation:
open parentheses, operation, arguments, close parentheses.
Therefore, in  \verb|c2f|,  \verb|(+ x 40)|
adds \verb|x| to 40, and \verb|(* (+ x 40) 9)| multiplies $x+40$ by 9.

In order to perform a few tests, Nia must initially save
the program with \verb|C-x C-s|. Then she 
presses \verb|C-x 2| to create a window for interacting
with Lisp. After this action, there are two
windows on the screen, both 
showing the \verb|celsius.scm| buffer. The \verb|C-x o| command
makes the cursor jump from one window to the other.
From the bottom window, Nia issues the  \verb|M-x shell| command
to create a Read Eval Print Loop (she maintains the
\keys{Alt} key down and presses \verb|x|; then she types
the \verb|shell| command on the minibuffer). 

From the shell buffer, Nia can launch the Scheme compiler,
and load the \verb|celsius.scm| program. Then she can
call any application that she has define herself, or which
comes out of the box with Lisp. The Read Eval Print Loop,
or repl for short, does what its name indicates: reads a
command, evaluates it, prints the result and loops back
to the next command. The first operation is loading the
\verb|celsius.scm| file, as you can see in
figure~\ref{fig:repl}. Be sure that you are on the repl
interaction window when you perform the loading, otherwise
all you will get are error messages from the shell.

\begin{figure}[!h]
	\begin{fmpage}{0.8\linewidth}
		\begin{verbatim}
		; File: celsius.scm

		(define (c2f x)
   (- (/ (* (+ x 40) 9) 5.0) 40))

   (define (f2c x)
    (- (/ (* (+ x 40) 5.0) 9) 40))

		\end{verbatim}
	\end{fmpage}

	\begin{fmpage}{0.8\linewidth}
		\begin{verbatim}
		› scheme
		Chez Scheme Version x.y.z

		> (load "celsius.scm")
		> (c2f 100)
		212.0
		\end{verbatim}
	\end{fmpage}
	\begin{fmpage}{0.8\linewidth}
		\verb|M-x shell|
	\end{fmpage}
	\caption{Source file and Interation Buffer}
	\label{fig:repl}
\end{figure}

Let us recall what happened until now. Nia pressed
\verb|C-x 2| to split  the buffer window and execute the
\verb|M-x shell| command, as shown in figure~\ref{fig:repl}.
In my version of Emacs, the \verb|M-x shell| command alone
splits the window and launch a shell terminal. I don't
need the \verb|C-x 2| step.

Eventually, Nia has two buffers in front of her, each in
its own window. One of the buffers has a Lisp source file,
while the other contains a Lisp repl interaction. In order
to  switch from one buffer to the other, Nia press the
\verb|C-x o| command. The command \verb|C-x C-s| saves the
Lisp source file.  The command \verb|(load "celsius.scm")|
from the repl loads the  source file.

After loading the source file, Nia types \verb|(c2f 100)|
on the interaction buffer. Then she presses the
\keys{Enter} key to perform the calculation of the Fahrenheit
reading and print the value found.

Nia often works with a single window.  In this case, after
typing and saving the source file with \verb|C-x C-s|, she
does not split the window before opening the interaction
buffer with the \verb|M-x shell| command. In this case,
only one buffer is visible. Nia switches  between the
interaction buffer and the source file by pressing the
\verb|C-x b| command and choosing the destination buffer
with the arrow \keys{~$\uparrow$~} \keys{~$\downarrow$~}
keys.

Text editing is one of these things that are easier to do
than to explain. Therefore, the reader is advised to learn
Emacs by experimenting with the commands.

\section{let-binding}\index{let-binding}
The let-form introduces\index{Local variables} 
local variables\index{Local variables!let-binding}
through a list of \verb|(id value)| 
pairs.\index{let-binding!basic let-binding}
In the basic let-binding, a variable cannot 
depend on previous values that appear
in the local list. In the\index{let-binding!let-star binding}
\verb|let*| (let star) binding,
one can use previous bindings to calculate
the value of a variable, as one can see in
figure~\ref{fig:date-calculation}.
When using a let binding, the programmer must
bear in mind that it needs parentheses for
grouping together the set of \verb|(id value)| pairs,
and also parentheses for each pair. Therefore,
one must open two parentheses in front of the
first pair, one for the list of pairs and the
other for the first pair.


You certainly know that the word `December' means
the tenth month; it comes from the Latin word for
ten, as attested in words such as:
\begin{itemize}
	\item {\em decimal} -- base ten.
	\item {\em decimeter} -- tenth part of a meter.
	\item {\em decimate} -- the killing of every tenth
		Roman soldier that performed badly in battle.
\end{itemize}

October is named after the Latin numeral for {\em eighth}.
In fact, the radical {\sc Oct-} can be translated
as {\em eight}, like in
{\em octave}, {\em octal}, and {\em octagon}.
One could continue with this analysis,
placing September in the seventh,
and November in the ninth position
in the sequence of months.
But everybody and
his brother know that December is the
twelfth, and that
October is the tenth month of the
year. Why did the Romans give misleading
names to these months?

Rome was purportedly founded by Romulus,
who designed a calendar.
March was the first month of the year
in the Romulus calendar. Therefore, if Nia
wants the order for a given month in
this mythical calendar, she must
subtract 2 from the order
in the Gregorian calendar.
In doing so, September becomes the
seventh month; October, the eighth;
November, the ninth and
December, therefore, the tenth month.

\begin{figure}[!h]
	\begin{fmpage}{0.9\linewidth}
		\begin{verbatim}
		(define (winter m) (quotient (- 14 m) 12))

		(define (roman-order m) (+ m (* (winter m) 12) -2))

		(define (zr y m day)
		(let* [ (roman-month (roman-order m))
	(roman-year (- y (winter m)))
	(century (quotient roman-year 100))
	(decade (remainder roman-year 100)) ]
 (mod (+ day
	 (quotient (- (* 13 roman-month) 1) 5)
	 decade
	 (quotient decade 4)
	 (quotient century 4)
	 (* century -2)) 7)) )
		\end{verbatim}
	\end{fmpage}

	\begin{fmpage}{0.9\linewidth}
		\verb|> (load "zeller.scm")|\\
		\verb|> (zr 2016 8 31)|\\
		3
	\end{fmpage}
	\caption{Source file and Interaction Buffer}
	\label{fig:date-calculation}
\end{figure}

Farming and plunder were the main
occupations of the Romans, and since winter
is not the ideal season for
either of these activities,
the Romans did not care much for
January and February. However,
Sosigenes of Alexandria,
at the request of Cleopatra and
Julius C\ae sar, designed the
Julian calendar, where January
and February, the two deep winter
months, appear in  positions
1 and 2 respectively.
Therefore, a need for a formula
that converts from the modern
month numbering to the
old sequence has arisen.
\begin{quote}
	\begin{verbatim}
	(define (roman-order m) (+ m (* (winter m) 12) -2))
	\end{verbatim}
\end{quote}
The above formula will work if \verb|(winter m)|
returns 1 for months 1 and 2, and
0 for months from 3 to 12. The definition
below satisfies these requisites.
\begin{quote}
	\begin{verbatim}
	(define (winter m) (quotient (- 14 m) 12))
	\end{verbatim}
\end{quote}
For months from 3 to 12,
\verb|(- 14 m)| produces a number
less than 12, and \verb|(quotient (- 14 m) 12)|
is equal to 0. Therefore,
\verb|(* (winter m) 12)| is equal to
0 for all months from March through December,
and the conversion formula reduces
to \verb|(+ m -2)|, which means that
March will become the Roman month 1, and December
will become the Roman month 10.
However, since January is month 1,
and February is month 2 in the Gregorian
calendar, \verb|(winter m)| is
equal to 1 for these two months.
In the case of month 1,
the Roman order is given by
\verb|(+ 1 (* 1 12) -2)|, that is
equal to 11. For month 2, one has that
\verb|(+ 2 (* (winter 2) 12) -2)|
is equal to 12.

The program of figure~\ref{fig:date-calculation}
calculates the day of the week
through Zeller's congruence. In the definition
of \verb|(zr y m day)|, the let-star 
binds values to the local variables \verb|roman-month|,
\verb|roman-year|, \verb|century| and \verb|decade|.

Figure~\ref{fig:date-calculation} shows that
the let binding avoids recalculating values that
appear more than once in a program. This
is the case of \verb|roman-year|, that
appears four times in the \verb|zr| procedure.
Besides this, by identifying important
subexpressions with meaningful names, the
program becomes clearer and cleaner.
After loading the program of
figure~\ref{fig:date-calculation},
expressions
such as
\verb|(zr 2016 8 31)| return
a number between 0 and 6, corresponding to
Sunday, Monday, Tuesday, Wednesday,
Thursday, Friday and Saturday.

\section{True or False}\index{Conditional execution!True and False}
So far, the only tool that Nia has used
for programming is combination of functions.
However, this is not enough to write
programs that reach a decision or
make a choice. Scheme has expressions
to say ``if this is true, do one thing,
else do another thing''. Nia 
could have used one of these decision
making expressions to calculate the roman
order:
\begin{quote}
	\begin{verbatim}
	(if (< m 3) (+ m 10) (- m 2))
	\end{verbatim}
\end{quote}\index{Conditional execution!if-else}
The if-else expression says: if the
Julian numbering \verb|m| is greater
than 2, then subtract 2 from \verb|m|,
else add 10 to \verb|m|. The \verb|if-else|
expression is much simpler to understand
than all that confusing talk about deep
winter months.

Besides correcting the month, the \verb|(winter m)|
expression was used to correct the year. If
the month \verb|m| is 1 or 2, one must subtract
1 from the Gregorian year in order to obtain
the Roman year.

You probably know, in all computer
languages, {\em Boolean} is a data type
that can have just two values: \verb|#t|
for ``true'' and \verb|#f| for ``false''.
The procedure \verb|(< m 3)| returns
\verb|#t| if \verb|m| is less than 3,
and \verb|#f| otherwise. On the other
hand, \verb|(if (< m 3) (+ m 10) (- m 2))|
calculates the expression \verb|(+ m 10)|
if and only if \verb|(< m 3)| produces
the value \verb|#t|. Otherwise, the
\verb|if| form calculates the \verb|(- m 2)|
expression.

\begin{figure}[!h]
	\begin{fmpage}{0.9\linewidth}
		\begin{verbatim}
		(define (zr y m day)
		(let* [ (roman-month (if (< m 3) (+ m 10) (- m 2)))
	(roman-year (if (< m 3) (- y 1) y))
	(century (quotient roman-year 100))
	(decade (remainder roman-year 100)) ]
 (mod (+ day decade
	 (quotient (- (* 13 roman-month) 1) 5)
	 (quotient decade 4)
	 (quotient century 4)
	 (* century -2)) 7)) )
		\end{verbatim}
	\end{fmpage}

	\begin{fmpage}{0.9\linewidth}
		\begin{verbatim}
		> (load "zeller.scm")
		#<function>
		> (zr 2016 8 31)
		3
		\end{verbatim}
	\end{fmpage}
	\caption{Making decisions}
	\label{fig:zeller}
\end{figure}

Now, let us analyze the terms of the Zeller's
congruence. A
normal year has 365 days and 52 weeks.
Since 365 is equal to \verb|(+ (* 7 52) 1)|,
each normal year advances the day of the
week by 1. Therefore, the formula has a
\verb|decade| component.
Every four years
February has an extra day,
so it is necessary to add
\verb|(quotient decade 4)|
to the formula.

A century contains 100 years.
Therefore, one would expect  25
leap years in a century. But since
turn of the century years, such as 1900, are
not leap years, the number of leap
years in a century reduces to 24.
So, in a century the day of the
week advances by 124. But
\verb|(remainder 124 7)| produces
5, that is equal to \verb|(- 7 2)|.
Then one needs to subtract 2 for
each century. This is done through
the term \verb|(* century -2)|.

But every fourth century year is
a leap year. Therefore, Zeller needed to 
add the term \verb|(quotient century 4)|
to the formula.

Starting from March, each month
has 2 or 3 days beyond 28, that is
the approximate duration of a lunar
month: 3, 2,
3, 2, 3, 3, 2, 3, 2, 3, 3, 0.
Nia noticed that definition
\begin{quote}
	\verb|(define (acc i) (- (quotient (- (* 13 i) 1) 5) 2))|
\end{quote}
returns the accumulated month contribution
to the week day.
However, the day chosen to start
the week cycle is irrelevant. Therefore,
there is no need to subtract 2 from the
accumulator.

\chapter{Sets}
A set is a collection of things. You certainly know that
collections do not have repeated items. I mean, if a guy or gall has
a collection of stickers, s/he does not want to have two
copies of the same sticker in his/her collection. If s/he has a
repeated item, s/he will trade it for another item that s/he lacks
in his/her collection.


It is possible that if your father has a collection of Greek
coins, he will willingly accept another drachma in his set.
However, the two drachmas are not exactly equal; one of them may be older
than the other.

\section{Sets of numbers}
Mathematicians collect other things, besides
coins, stamps, and slide rules. They collect numbers, for instance;
therefore you are supposed to learn a lot about sets of numbers.
\begin{description}
	\item[$\mathbb{N}$ is the set of natural integers.] Here is how mathematicians  write the
		elements of $\mathbb{N}$: $\{0,1,2,3,4\ldots\}$.
	\item[$\mathbb{Z}$ is the set of integers, i.e.,] $\mathbb{Z}=\{\ldots-3, -2, -1, 0, 1, 2, 3, 4\ldots\}$.
\end{description}

%%\begin{wrapfigure}[10]{i}{4cm}
%%\includegraphics[scale=0.4]{figs/cantor.png}
%%\caption{Cantor}
%%\end{wrapfigure}

Why is the set of integers represented by the letter $\mathbb{Z}$? I do not know,
but I can make an educated guess. The set theory was discovered by Georg Ferdinand
Ludwig Philipp Cantor, a Russian whose parents were Danish, but who  wrote
his {\em Mengenlehre} in German! In German, integers may have some strange name like {\em Zahlen}.


You may think that set theory is boring; however, many people think that it is quite interesting. For instance, there is an Argentinean that scholars consider to be the greatest writer that lived after the fall of Greek civilization. In few words, only the Greeks could put forward a better author. You probably heard Chileans saying  that Argentineans are somewhat conceited. {\em You know what is the best possible deal? It is to pay a fair price for Argentineans, and resell them at what they think is their worth}. However, notwithstanding the opinion of the Chileans, Jorge Luiz Borges is the greatest writer who wrote in a language different from Greek.
Do you know what was his favorite subject? It was the Set Theory, or {\em Der Mengenlehre}, as he liked to call it.

When a mathematician wants to say that an element is a member of a set, he writes something like
$$3  \in \mathbb{Z}$$
If he wants to say that something is not an element of a set, for instance, if he wants to state that $-3$ is not an element of $\mathbb{N}$, he writes:
$$-3  \notin  \mathbb{N}$$
Let us summarize the notation that Algebra teachers use, when they explain set theory to their students.
\begin{description}
	\item[Vertical bar.] The weird notation $\{x^2 | x \in \mathbb{N}\}$
		represents the set of
		$x^2$, {\em such that} $x$ is a member of $\mathbb{N}$, or else,
		$\{0, 1, 4, 9, 16, 25\ldots\}$
	\item[Conjunction.] In Mathematics, you can use a symbol $\wedge$
		to say {\bf\small and}; therefore $x>2 \wedge x<5$ means 
		\verb|(and (> x 2) (< x 5))| in Lisp notation.
	\item[Disjunction.] The expression
		$(x<2) \vee (x>5)$
		means \verb|(or (< x 2) (> x 5))| in Lisp notation
\end{description}
Using the above notation, you can define the set of rational numbers:
$$\mathbb{Q}=\{\frac{p}{q} | p \in \mathbb{Z} \wedge q \in \mathbb{Z} 
\wedge q \neq 0\}$$
In informal English, this expression means that a rational number is a fraction $$\frac{p}{q}$$ such that $p$ is a member of $\mathbb{Z}$ and $q$ is also a member of $\mathbb{Z}$, submitted to the condition that $q$ is not equal to $0$.

In femtolisp, one can represent a rational number
as a Cartesian pair of integers. The expression \verb|(cons p q)|,
where \verb|p| and \verb|q| are integers, builds such a pair.
For instance, \verb|(cons 5 3)| produces the
\verb|'(5 . 3)| pair. N.B. there
are spaces before and after the dot.\label{page:cartesian-pair}

In the dotted pair notation of a rational number,
the \verb|(car '(5 . 3))| operation retrieves
the first element of the pair, i.e., 5.
On the other hand, the 
\verb|(cdr '(5 . 3))| operation
returns the second element, which is 3.

\begin{figure}[!h]
	\begin{fmpage}{0.9\linewidth}
		\begin{verbatim}
		(define f (cons 2 3))

		(define g (cons 4 5))

		(define (add x y)
   (let [ (numerator (+ (* (car x) (cdr y))
			(* (car y) (cdr x)) ))
	  (denominator  (* (cdr x) (cdr y)))]
      (cons numerator
	    denominator))) 
		\end{verbatim}
	\end{fmpage}

	\begin{fmpage}{0.9\linewidth}
		\verb|> (load "cartesian.scm")|\\
		\verb|> (add f g)|\\
		\verb|(22 . 5)|
		\verb|(add '(3 . 7) '(2 . 5))|\\
		\verb|(29 . 35)|
	\end{fmpage}
	\caption{Cartesian pairs}
	\label{fig:cartesian-pairs}
\end{figure}

Let us assume that Nia wants to add two 
rational numbers. From elementary arithmetic,
Nia knows that the addition of two rational numbers
$P$ and $Q$ is given by the following expression:
\begin{equation}
	\frac{X_a}{X_d}+\frac{Y_a}{Y_d}=
	\frac{X_a\times Y_d + Y_a\times X_d}{X_d\times Y_d}
	\label{eq:addratnums}
\end{equation}
In the dotted pair notation, one has $X_a$= \verb|(car X)|,
$X_d=\verb|(cdr X)|$, $Y_a$= \verb|(car Y)| and
$Y_d$= \verb|(cdr Y)|. By replacing these values
in equation~\ref{eq:addratnums}, one arrives at the
following expressions for the numerator 
and the denominator of the addition:
\begin{quote}
	\begin{verbatim}
	(+ (* (car x) (cdr y))    ;; numerator
   (* (car y) (cdr x)))   

   (* (cdr x) (cdr y))       ;; denominator
	\end{verbatim}
\end{quote}
Nia used the above expressions for 
calculating the numerator
and denominator of $X+Y$ in 
listing~\ref{fig:cartesian-pairs}.


\section{Irrational Numbers}
At Pythagora's time, the Greeks claimed that
any pair of line segments is commensurable, i.e.,
you can always find a meter,
such that the lengths of any two segments
are given by integers.  The following
example will
show how the Greek theory of commensurable lengths at work.
Consider the square that the Greek in the figure
at the bottom of this page is evaluating.


If the Greeks were right, I would be able to find a meter, possibly a very small one,
that produces an integer measure for the diagonal of the square,
and another integer measure for the side. Suppose that $p$ is the
result of measuring the side of the square, and $q$ is the result of measuring the diagonal.
The Pythagorean theorem states that
$\overline{AC}^2+\overline{CB}^2= \overline{AB}^2$, i.e.,
\begin{equation}
	p^2+p^2= q^2\therefore 2p^2=q^2\label{Pytagoras1}
\end{equation}

\begin{wrapfigure}[10]{i}{4cm}
	\includegraphics{fig-sets/apolob.png}
\end{wrapfigure}
You can also choose the meter so that $p$ and  $q$ have no common factors. For instance,
if both $p$ and $q$ were divisible by 2, you could double the length of the meter, getting
values no longer divisible by 2. E.g. if $p=20$ and $q=18$, and you double the length of
the meter, you get $p=10$, and $q=9$. Thus let us assume that one has chosen a meter so
that $p$ and $q$ are not simultaneously even. But from equation~\ref{Pytagoras1}, one
has that $q^2$ is even. But if $q^2$ is even, $q$ is even too. You can check that
the square of an odd number is always an odd number. Since $q$ is even, you can
substitute $2\times n$ for it in equation~\ref{Pytagoras1}.
\begin{equation}
	2\times p^2= q^2= (2\times n)^2= 4\times n^2 \therefore 2\times p^2= 4\times q^2
	\therefore p^2= 2\times n^2
	\label{peven}
\end{equation}

\begin{wrapfigure}[11]{o}{5.5cm}
	\begin{center}
		\includegraphics[scale=0.5]{fig-sets/apoloa.png}
	\end{center}
\end{wrapfigure}
Equation~\ref{Pytagoras1} shows that $q$ is even;
equation~\ref{peven} proves that $p$ is even too. But this is against our
assumption that $p$ and $q$ are not both even. Therefore, $p$ and $q$ cannot be
integers in equation~\ref{Pytagoras1}, which you can rewrite as
$$\frac{p}{q}=\sqrt{2}$$



The number $\sqrt 2$, that gives the ratio
between the side and the diagonal of any square, is not an element of $\mathbb{Q}$,
or else, $\sqrt 2 \notin  \mathbb{Q}$. It was Hypasus of Metapontum, a Greek philosopher,
who proved this. The Greeks were a people of wise men and women. Nevertheless they
had the strange habit of consulting with an illiterate peasant girl at Delphi,
before doing anything useful. Keeping with this tradition, Hyppasus  asked
the Delphian priestess--- the illiterate girl--- what he should do to please Apollo.
She told him to measure  the side and the diagonal of the god's square
altar using the same meter. By proving
that the problem was impossible, Hypasus discovered a type of number that can not be
written as a fraction. This kind of number is called irrational.

An irrational number
is not a crazy, or a stupid number; it is simply a number that you cannot represent
as {\em ratione} (fraction, in Latin).

The set of all numbers, integer, 
irrational, and rational is called $\mathbb{R}$, or
the set of real numbers. 

Computers are not able to deal
with sets that hold a transfinite number of
elements. Therefore, femtolisp and
all other programming languages
replace sets with the concept of type.
In femtolisp, $\mathbb{Z}$ is 
called {\bf integer}, although
the {\bf integer} type does not cover all integers, 
but enough of them to satisfy
your needs. Likewise, a floating point
number belongs to the {\bf number} type, 
which is a subset of $\mathbb{R}$.


If $x\in{\bf Int}$, Lisp programmers\index{Type} say 
that $x$ has 
type\label{type:definition} \index{Type!Int} {\bf integer}. 
They also say that $r$ has 
type {\bf number} if $r\in {\bf Real}$.

In femtolisp, rational and irrational numbers
are inexact data types. Here are a few functions
that produce inexact results:
\begin{quote}
	$\verb|(+ |x_1\;x_2\;\ldots x_n\verb|)|$ -- addition\\
	$\verb|(* |x_1\;x_2\;\ldots x_n\verb|)|$ -- multiplication\\
	$\verb|(- |x_1\;x_2\;\ldots x_n\verb|)|$ -- subtraction\\
	$\verb|(/ |x_1\;x_2\;\ldots x_n\verb|)|$ -- division\\
	$\verb|(pow |x\;y\verb|)|$ -- $x^y$\\
	$\verb|(sin | x \verb|)|$ -- $\sin(x)$\\
	$\verb|(asin | x \verb|)|$ -- $\arcsin(x)$\\
	$\verb|(cos | x \verb|)|$ -- $\cos(x)$\\
	$\verb|(acos | x \verb|)|$ -- $\arcsin(x)$\\ 
	$\verb|(tan | x \verb|)|$ -- $\tan(x)$\\
	$\verb|(atan | x \verb|)|$ -- $\arctan(x)$\\
	$\verb|(log | x \verb|)|$ -- natural logarithm\\
	$\verb|(log2 | x \verb|)|$ -- logarithm in base 2\\
	$\verb|(log10 | x \verb|)|$ -- logarithm in base 10
\end{quote}\index{Predicates!integer?}
Integers are distinguished from inexact numbers
by the \verb|(integer? x)| predicate. Two important
integer functions are \verb|(quotient x y)| and
\verb|(mod x y)|. The first function
produces the integer division of \verb|x|
by \verb|y|, while the other returns
the remainder of the division. You should remember
having used
these two functions earlier to calculate the Zeller's
congruence.

\section{Data types}
There are other types
besides {\bf integer}, and {\bf number}. 
Here is a  list of primitive types:
\begin{description}
	\item[integer ---] Integer numbers between $-2147483648$ and $2147483647$. \index{Predicates!exact?}
		The \verb|(exact? x)| function returns \verb|#t| if \verb|x|
		is an integer, and \verb|#f| otherwise.
	\item[inexact ---] the inexact type represents both
		rational and irrational numbers, albeit approximately.
		The \verb|(inexact? x)| function returns \verb|#t| for
		inexact numbers.\index{Predicates!inexact?}
	\item[number ---] A number can be either exact or inexact.
		The \verb|(number? x)| function returns \verb|#t| for numbers,
		and \verb|#f| for any other type of object.\index{Predicates!number?}
	\item[String ---] A quoted\index{String} string \index{Type!String}
		of characters: \verb|"3.12"|, \verb|"Hippasus"|, \verb|"pen"|, etc.
		The \verb|(string? x)| function answers \verb|#t| 
		if \verb|x| is a string, 
		and \verb|#f| otherwise.\index{Predicates!string?}
		A few important functions that operate on strings are:
		\begin{itemize}
			\item \verb|(string-length s)| returns the number
				of chars of the \verb|s| string.
			\item\verb|(substring "Hello, World" 2 5)| operation
				returns the \verb| "llo"| substring.
				\verb|(substring "Hello, World" 0 4)| returns the
				\verb|"Hell"| substring. And so on.
		\end{itemize}
	\item[Char ---] Chars have the \verb|#\|
		prefix: \index{Type!Char}
		\verb|#\A|,
		\verb|#\b|, \verb|#\3|, \verb|#\space|,
		\verb|#\newline|, etc.
		The \verb|(char? x)| function returns \verb|#t|
		when \verb|x| is a char.\index{Predicates!char?}
		Let \verb|s| be
		a string such as \verb|"Hello"|. Then,
		\verb|(string-ref "Hello" 0)|
		returns the first char of \verb|s|,
		\verb|(string-ref "Hello" 1)| returns
		the second char, and so on. One can compare
		chars with the following functions:
		\begin{quote}
			\verb|(char=? x #\g)| -- \verb|#t| if \verb|x= #\g|\\
			\verb|(char>? x #\g)| -- \verb|#t| if \verb|x| comes
			after the \verb|#\g| char.\\
			\verb|(char<? x #\g)| -- \verb|#t| if \verb|x| comes
			before the \verb|#\g| char.\\
			\verb|(char>=? x #\g)| -- \verb|#t| if \verb|x| is
			equal to \verb|#\g| or comes after it.\\
			\verb|(char<=? x #\g)| -- \verb|#t| if \verb|x| is
			equal to \verb|#\g| or comes before it.\\
		\end{quote}
	\item[Pair ---] One can use a pair data type
		both to represent Cartesian pairs and lists.
		The \verb|(pair? x)| returns \verb|#t| if
		\verb|x| is a pair,\index{Predicates!pair?}
		otherwise it returns \verb|#f|.
		The \verb|'()| empty list is a very important
		object that one can use as the second element
		of a pair. The \verb|(null? x)| returns \verb|#t|
		when \verb|x| is the \verb|'()| empty list.
\end{description}


\section{Functions}
A function is a relationship between an argument and a unique value.
Let the argument be $x\in B$, where $B$ is a set; then $B$ is called
domain of the function. Let the value be $f(x)\in C$, where $C$ is
also a set; then $C$ is the range of the function.
Functions can be represented by tables, or clauses. Let us examine
each one of these representations in turn.

\subsection*{Tables}
Let us consider a function that associates \verb|#t| ({\em true})
or \verb|#f| ({\em False}) to the
letters of the Roman alphabet. If a letter is a vowel,
then the value will be \verb|#t|; otherwise, 
it will be \verb|#f|. The range of such a function
is \verb|{#t, #f}|, and the domain is
$\{a,b,c,d,e,f,g,h,i,j,k,l,m,n,o,p,q,r,s,t,u,v,w,x,y,z\}$.\\

\verb||\\
{\footnotesize
\begin{tabular}{|p{1.2cm} p{1.2cm} | p{1.2cm} p{1.2cm} |p{1.2cm} p{1.2cm} | p{1.2cm} p{1.2cm} |}
	Domain & Range & Domain & Range & Domain & Range & Domain & Range\\
	a & \verb|#t| & g &\verb|#f| & m &\verb|#f| & t & \verb|#f| \\
	b & \verb|#f| & h &\verb|#f| & n & \verb|#f| & u & \verb|#t| \\
	c & \verb|#f| &  i &\verb|#t| &o &\verb|#t| & v & \verb|#f|\\
	d &\verb|#f| &  j & \verb|#f| & p & \verb|#f| & w & \verb|#f| \\
	e &\verb|#t| & k & \verb|#f| & q & \verb|#f| & x & \verb|#f| \\
	f &\verb|#f| &  l &\verb|#f| & r & \verb|#f| & y & \verb|#f| \\
	&            &                 &  & s & \verb|#f| & z & \verb|#f|\\
\end{tabular}}

\subsection{Clauses}
From what you have seen in the last section, you certainly notice that it is pretty tough to represent a
function using a table. You must list every case. There are also functions,
like $\sin\;x$, whose domain has an infinite number of values, which makes
it impossible to list all entries. 
Even if you were to try to insert only a finite
subset of the domain into the table, it wouldn't be easy. 
Notwithstanding, in the past,
people used to build tables. In fact, tables were the only way to calculate many
useful functions, like $(\sin\;x)$, 
$(\log\;x)$, $(\cos\;x)$, etc. In 1814 Barlow
published his Tables which give factors, squares, cubes, square roots, reciprocals
and hyperbolic logs of all numbers from 1 to 10000. In 1631 Briggs published
tables of sin functions to 15 places and tan and sec functions to 10 places.
I heard the story of a mathematician who published a table of sinus, and made
a mistake. Troubled by the fact that around a hundred sailors lost their way due to his
mistake, that mathematician committed suicide. This story shows that the use
of tables may be hazardous to your health.

In order to understand how to represent a function with clauses, let us revisit the vowel table.
Using clauses, that table becomes
\begin{quote}\small
	\begin{verbatim}
	(define (vowel x)
   (cond [ (or (char=? x #\a)(char=? x #\e)(char=? x #\i) 
	   (char=? x #\o)(char=? x #\u)) #t]
	 [else #f]))
	\end{verbatim}
\end{quote}
Expressions like $(\verb|or | p_1\;p_2\ldots p_n)$ has the same 
meaning as $p_1 \vee p_2\ldots \vee p_n$.

Functions have a parameter, also called variable, that represents an element of the domain.
Thus, the vowel function has a parameter \verb|x|, that represents an element of the set
\verb|{'a','b','c','d','e','f',...}|.
Below the name of the function, and its variable, one finds a set of clauses.
Each clause has a condition followed by an expression, that gives the value,
if that clause applies. Consider the first clause. The expression:
\begin{quote}
	\begin{verbatim}
	(or (char=? #\a)(char=? #\e)(char=? #\i)
    (char=? #\o)(char=? #\u))
	\end{verbatim}
\end{quote}
asks the question: {\em Is \verb|x= #\a|,
\verb|x= #\a|, \verb|x= #\a|, \verb|x= #\a| or
\verb|x= #\a|?} If the answear is yes,
the  clause  value is \verb|#t|; in proper functions,
the clause
value is given by the second clause expression. 

Now, let us consider the Fibonacci function,
that has such an important role in the
book ``The Da Vinci Code''.
Here is its table for the first 9 entries:\\
\begin{quote}\label{page:Fibonacci}
	\begin{tabular}{p{1cm}p{3cm} | p{1cm}p{3cm} | p{1cm}p{3cm}}
		0 & 1 & 3 & 5 & 6 & 21\\
		1 & 2 & 4 & 8 & 7 & 34 \\
		2 & 3 & 5 & 13 & 8 & 55\\
	\end{tabular}
\end{quote}
Notice that a given functional value is equal to
$f_{n}= f_{n-3}+2\times f_{n-2}$. Assume that $n=6$.
Then,  $f_6= f_3+2\times f_4$.

Of course, the expression $f_{n}= f_{n-3}+2\times f_{n-2}$
is true only for $n>2$, since there are not three precedent
values for $n=0$, $n=1$ and $n=2$.  The below program shows
how  you can state that a rule is valid only under certain
conditions.\\

\begin{fmpage}{0.9\linewidth}
	\begin{verbatim}
	(define (fib n)
   (cond [ (< n 2) (+ n 1)]
	 [ (= n 2) 3]
	 [else  (+ (fib (- n 3))
		   (* (fib (- n 2)) 2))]))
	\end{verbatim}
\end{fmpage}

\begin{fmpage}{0.9\linewidth}
	\begin{verbatim}
	> (load "fibonacci.scm")
	> (fib 50)
	32951280099
	\end{verbatim}
\end{fmpage}



\paragraph{Predicates.} A predicate
is a function that gives
true and false for output. In Scheme,
the only false value is \verb|#f|, but
any value that is different from \verb|#f|
is considered true. A predicate can be used to discover
whether a property is true for a given value.
For instance, a sequence of elements
such as \verb|'(S P Q R)| is called
list. There is also an empty list,
that has no elements at all, and
is represented by \verb|'()|. Let
\verb|xs| be a list. Then, the predicate
\verb|(null? xs)| returns \verb|#t|
if \verb|xs| is the empty list.

There are predicates designed for
performing comparisons. For instance,
\verb|(> m 2)| returns \verb|#t| for
\verb|m| greater than 2, and \verb|#f|
otherwise. Here is a more or less complete
list of comparison predicates:
\begin{itemize}
	\item\verb|(> m n)| -- \verb|#t| for
		\verb|m| greater than \verb|n|.
	\item\verb|(< m n)| -- \verb|#t| for
		\verb|m| less than \verb|n|.
	\item\verb|(>= m n)| -- \verb|#t| for
		\verb|m| greater or equal to \verb|n|.
	\item\verb|(<= m n)| -- \verb|#t| for
		\verb|m| less or equal to \verb|n|.
	\item\verb|(= m n)| -- \verb|#t| if
		\verb|m| is a number equal to \verb|n|.
\end{itemize}
The inputs \verb|m| and \verb|n| must
be both numbers, if you want
the above predicates to work.


A string is a sequence of characters
represented between double quotation
marks. For instance, \verb|"Sunday"|
is a string. Below, there is a short
list of string predicates.
\begin{itemize}
	\item\verb|(string=? s z)| -- \verb|#t|
		if \verb|s| and \verb|z| are equal.
	\item\verb|(string>? s z)| -- \verb|#t|
		if \verb|s|  comes
		after \verb|z| in the alphabetical order.
	\item\verb|(string<? s z)| -- \verb|#t|
		if \verb|s| comes before \verb|z|
		in the alphabetical order.
	\item\verb|(string<=? s z)| -- \verb|#t| if
		\verb|s| precedes or is equal to \verb|z|.
	\item\verb|(string>=? s z)| -- \verb|#t| if
		\verb|s| follows or is equal to \verb|z|.
\end{itemize}

The prefix \verb|#\| identifies isolated
characters. Let us assume the following
definition:
\begin{quote}
	\begin{verbatim}
	(define s "Nia Vardalos")
	\end{verbatim}
\end{quote}
Then, the characters
\verb|#\N| \verb|#\i| 
\verb|#\a| and \verb|#\space|
were retrieved from
\verb|s| by the
expressions \verb|(string-ref s 0)|,
\verb|(string-ref s 1)|,
\verb|(string-ref s 2)| and
\verb|(string-ref s 3)|,
\verb|| respectively.

The blank space character,  control
characters, non-graphic characteres
and all other non-printable characters
can be represented by
identifiers, such as \verb|#\space|,
\verb|#\tab| and \verb|#\newline|.

Character have their own set of
predicates, as one should expect.
\begin{itemize}
	\item\verb|(char=? #\A #\a)| -- is \verb|#f|
		since \verb|#\A| and \verb|#\a| are not equal.
	\item\verb|(char>? #\z #\a)| -- is \verb|#t|
		since \verb|#\z|  comes
		after \verb|#\a| in the alphabetical order.
	\item\verb|(char<? #\a #\z)| -- is \verb|#t|
		since \verb|#\a| comes before \verb|#\z|
		in the alphabetical order.
	\item\verb|(char<=? #\a #\c)| -- is \verb|#t| since
		the order of \verb|#\a| is less or equal to
		the order of \verb|#\c|.
	\item\verb|(char>=? #\a #\c)| -- \verb|#f| since
		the order of \verb|#\a| greater or equal to the
		order of \verb|#\c|.
\end{itemize}

Besides predicates, Scheme uses special forms
to make decisions. The $(\textrm{and}\; p_1\;
p_2\ldots p_n)$ form
evaluates the sequence $p_1$, $p_2$\ldots $p_n$
until it reaches $p_n$ or one of the previous
$p_i$ returns \verb|#f|. In fewer words,
the and-form stops at the first
$p_i$ that returns \verb|#f| and, if no
argument is false, it stops at $p_n$.
The form returns the value of the last
$p_i$ that it evaluates. The and-form
returns a value different from \verb|#f|
if and only if all $p_i$ predicates produce
values different from \verb|#f| (false).

Like the and-form, the or-form also stops
as soon as it can. In the case of the or-form,
this means returning true as soon as any
of the arguments is true. Remember that
true is anything different from \verb|#f|.
The  $(\textrm{or}\; p_1\;
p_2\ldots p_n)$ form returns the value
of the first true $p_i$ predicate.

The \verb|(not P)| form returns \verb|#t|
if \verb|P| produces \verb|#f|, and
evaluates to \verb|#f| if \verb|P| is true,
i.e., anything different from \verb|#f|.

With the and, or and not forms, Nia can
combine primitive functions to define
many interesting predicates. For instance,
the predicate \verb|(digit? d)| determines
wether \verb|d| is a digit.
\begin{quote}
	\begin{verbatim}
	(define (digit? d)
   (and (char>=? d #\0)
	(char<=? d #\9)))
	\end{verbatim}
\end{quote}
The predicate \verb|deep-winter?| checks if
\verb|m| is 1 or 2:
\begin{quote}
	\begin{verbatim}
	(define (deep-winter? m)
    (or (= m 1) (= m 2)))
	\end{verbatim}
\end{quote}


\section{The cond-form}\label{page:cond-form}
Now that Nia knows how to find
an integer between 0 and 6 for the 
day of the week, she needs a function
that produces the corresponding name.


\begin{figure}[!h]
	\begin{fmpage}{\linewidth}
		\begin{verbatim}
		(define (week-day n)
   (cond [ (= n 0) 'Sunday]
	 [ (= n 1) 'Monday]
	 [ (= n 2) 'Tuesday]
	 [ (= n 3) 'Wednesday]
	 [ (= n 4) 'Thursday]
	 [ (= n 5) 'Friday]
	 [ else 'Saturday]))

	 (define (zeller y m day)
    (let* [ (roman-month (if (< m 3) (+ m 10) (- m 2)))
	    (roman-year (if (< m 3) (- y 1) y))
	    (century (quotient roman-year 100))
	    (decade (remainder roman-year 100))]
       (week-day (mod (+ day decade
		   (quotient (- (* 13 roman-month) 1) 5)
		   (quotient decade 4)
		   (quotient century 4)
	     (* century -2)) 7)) )

		\end{verbatim}
	\end{fmpage}

	\begin{fmpage}{\linewidth}
		\verb|> (load "zeller.scm")|\\
		\verb|> (zeller 2018 8 31)|\\
		\verb|Friday|
	\end{fmpage}
	\caption{Day of the week}
	\label{fig:day-of-the-week}
\end{figure}

\index{Conditional execution!cond}
\index{cond!clauses}
The cond-form controls conditional
execution, based on a set of clauses.
Each clause has a condition followed by
a sequence of actions.  Lisp starts with the
top clause, and proceeds in decending order.
It executes the first clause whose 
condition produces a value different from \verb|#f|.
For instance, the first clause condition is
the \verb|(= n 0)| predicate.
If the predicate \verb|(= n 0)| 
returns \verb|#t| for the Sunday index,
the function \verb|week-day| returns \verb|'Sunday|.
If \verb|n= 1|, the second condition holds,
and the function produces the symbol \verb|'Monday|.
And so on.


\section{Lists}
In Lisp, there is a data structure called list,
that uses parentheses to represent nested
sequences of objects. One can use lists to
represent structured data.
For instance, in the snippet below,
\verb|xor-structure| shows
the structure of a combinatorial circuit
through nested lists.

\begin{verbatim}
(define *x* 42)

(define xor
(lambda(A B)
 (or (and A (not B))
     (and (not A) B)) ))

     (define xor-structure
     '(or (and A  
	(not B))
   (and (not A) 
	B)))
\end{verbatim}

Although Nia does not know what a combinatorial 
circuit is, she noticed that the structure
defined as \verb|xor-structure| is prefixed by a
single quotation mark, that in Lisp parlance is
known as quote. Since Scheme uses the
same notation for programs and structured
data, the computer needs a tag to set
data apart from code. Therefore, 
quote was chosen to indicate that
an object is a list, not a procedure that
needs to be executed by the computer.

The \verb|define| form creates global
variables. In the above source code,
the variable \verb|*x*| is an id for the
number 42,
while \verb|xor| is the id for a program
that calculates the output of a two
input exclusive or gate. Of course, Nia
could define the \verb|xor| gate as shown below:
\begin{verbatim}
(define *x* 42)

(define (xor A B)
 (or (and A (not B))
     (and (not A) B)) )

     (define xor-structure
     '(or (and A 
	(not B))
   (and (not A) 
	B)))
\end{verbatim}

From the examples, Nia discovered that there
are two ways of defining the \verb|xor| gate
as a combination of \verb|A| input, \verb|B| input,
\verb|and| gate, \verb|or| gate
and \verb|not| gate. In the first and
most popular style, one uses the
\verb|(xor A B)| format that is a mirror of
the gate application. The advantage of
this method is that it spares one
nesting level, and shows clearly how to use
the definition. 

In the second style of defining functions and
gates, the id of the abstraction 
appears immediately after the \verb|define|
keyword. The  arguments and the body of
the definition are introduced by a lambda form:
\label{page:lambda1}\index{lambda}

\begin{verbatim}
(define xor
(lambda(A B)
 (or (and A (not B))
     (and (not A) B)) ))
\end{verbatim}

This second style is quite remarkable because
the abstraction that defines the operation
is treated exactly like any other
data type existent in the language.
In fact, there is no formal difference
between the definition of \verb|*x*|
as an integer constant, and \verb|xor|
as a functional abstraction.

\begin{figure}[!h]
	\begin{fmpage}{0.8\linewidth}
		\begin{verbatim}
		; File: lists.scm

		(define xor-structure
		'(or (and A (not B))
    (and (not A) B)))
		\end{verbatim}
	\end{fmpage}

	\begin{fmpage}{0.8\linewidth}
		\begin{verbatim}
		> (load "lists.scm")
		> xor-structure
		(or (and A (not B)) (and (not A) B))
		> (car xor-structure)
		or
		> (cdr xor-structure)
		((and A (not B)) (and (not A) B))
		> (car (cdr xor-structure))
		(and A (not B))
		> (car (cdr (car (cdr xor-structure)) ))
		A
		> (car (cdr (cdr (car (cdr xor-structure)) )))
		(not B)
		\end{verbatim}
	\end{fmpage}

	\begin{fmpage}{0.8\linewidth}
		\verb||
	\end{fmpage}
	\caption{List selectors}
	\label{fig:selectors}
\end{figure}


Although Scheme represents programs and
data in the same way, it does not use
the same methods to deal with source
code and lists. In the case of source
code, the compiler translates it into
a virtual machine language that the
computer can easily and efficiently
process.

Data structures, such as lists, have
an internal representation with parts. 
In particular, a list is implemented
as a chain of pairs 
(vide page~\pageref{page:cartesian-pair}),
each pair containing
a pointer to a list element,
and another pointer to the next pair. 
Let us assume that
\verb|xs| points to the list
\verb|'(S P Q R)|.
This list corresponds to the following
chain of pairs:
\begin{quote}
	\begin{verbatim}
	[*|*]--->[*|*]--->[*|*]--->[*|*]--->()
	|        |        |        |        
	S        P        Q        R       
	\end{verbatim}
\end{quote}
The first pair has a pointer to \verb|S|,
and another pointer to the second pair.
The second pair contains pointers to 
\verb|P| and to the third pair. 
The third pair points to \verb|Q| and
to the fourth pair. Finally, the fourth
pair points to \verb|R| and to 
the \verb|()| empty list.

\begin{quote}
	\begin{verbatim}
	[*|*]---> next pair
	|        
	S      
	\end{verbatim}
\end{quote}
The operation \verb|(car xs)| produces
the first pointer of the \verb|xs| 
chain of pairs. In the case of
the \verb|'(S P Q R)| list, \verb|(car xs)|
returns \verb|S|. On the other hand,
\verb|(cdr xs)| yields the pair
after the one pointed out by \verb|xs|,
i.e., \verb|'(P Q R)|. A sequence
of \verb|cdr| applications permits the user to
go through the pairs of a list. \\

\begin{fmpage}{0.8\linewidth}
	\begin{verbatim}
	(define spqr
   '(S P Q R))
	\end{verbatim}
\end{fmpage}

\begin{fmpage}{0.8\linewidth}
	\begin{verbatim}
	> (load "spqr.scm")
	> spqr
	(S P Q R)
	> (cdr spqr)
	(P Q R)
	> (cdr (cdr spqr))
	(Q R)
	> (cdr (cdr (cdr spqr)))
	(R)
	\end{verbatim}
\end{fmpage}

\begin{fmpage}{0.8\linewidth}
	\verb||
\end{fmpage}

\vspace{0.5cm}
If all one needs is to represent
sequences, then contiguous memory
elements could be more
practical than pairs. However, as one
can see in figure~\ref{fig:selectors},
a list element can be a nested sublist.
In this case, 
the \verb|car| part of a pair can point down to
a sublist branch. The diagram below shows
that one can reach any part of a nested
list following a chain of \verb|car| and \verb|cdr|.
In such a chain, the \verb|cdr| operation
advances one pair along the list, and
the \verb|car| operation goes down into a sublist.
The \verb|cdr| operation is equivalent to a
right $\rightarrow$  arrow, while the \verb|car|
operation is acts like a down $\downarrow$ arrow.
\begin{quote}
	\verb"     "$\rightarrow$\verb"       "$\rightarrow$\\
	\verb"[*|*]cdr-[*|*]cdr---[*|*]--()"\\
	\verb" |        |         car"$\downarrow$\\
	\verb" |        |          |  " $\rightarrow$\\
	\verb" |        |         [*|*]cdr-[*|*]---[*|*]---()"\\
	\verb" |        |          |       car"$\downarrow$\verb"     |"\\
	\verb" |        |          |        |       |"\\
	\verb" |        |          |        |       B"\\
	\verb" |        |          |        |  "\\
	\verb" |        |         and      [*|*]---[*|*]---()"\\
	\verb" |        |                   |       |"\\
	\verb" |        |                   |       |"\\
	\verb" |        |                  not      A"\\
	\verb" |        |"\\
	\verb" or      [*|*]---[*|*]---[*|*]---()"\\          
	\verb"          |       |       |"\\                     
	\verb"         and      A      [*|*]---[*|*]---()"\\  
	\verb"                          |       |"\\                     
	\verb"                         not      B"                    
\end{quote}
The above example shows the chains and subchains of
pairs that one uses to represent the logic
circuit below:
\begin{quote}
	\begin{verbatim}
	(define xs
	'(or 
     (and  A  
	   (not B))
		    (and (not A)
			 B)))
	\end{verbatim}
\end{quote}

Let us assume that Nia wants to retrieve the
\verb|(NOT A)| part of the circuit. Considering
that each right $\rightarrow$ arrow is equivalent
to a \verb|cdr| operation, and each down $\downarrow$
arrow can be interpreted as a \verb|car| operation,
she must perform \verb|(car (cdr (car (cdr (cdr xs)) )))|
to reach the goal.


\section{The list constructor}\index{List!cons}
Since the \verb|car| and \verb|cdr| operations
select the two parts of a pair, they are called
{\em selectors}.
Besides the two selectors, lists have a
constructor: The operation \verb|(cons x xs)| builds a pair
whose first element is \verb|x|, and the 
remaining elements are grouped in \verb|xs|.

\begin{quote}
	\verb|> (cons 'and '(A B))|\keys{Enter}\\
	(and A B)
\end{quote}

One has learned previously that lists must be
prefixed by the special quote operator,
in order to differentiate them from programs.
To make a long story short, quote  prevents
the evaluation of a list or symbol.

When you first heard about pairs on
page~\pageref{page:cartesian-pair},
it was stated that the first element
of a pair is separated from the 
second by a dot. However, the dot
can be omitted if the second element
is itself a cartesian pair or
the empty list. Then,
the  much neater 
\verb|'(3 4)| list syntax is equivalent 
to the dotted Cartesian \verb|'(3 . (4 . ()))| pair.

A backquote, not to be confused with quote, also prevents
evaluation, but the backquote transforms the list 
into a template.\index{List!backquote}
When there appears a comma in the 
template, Lisp evaluates the expression following
the comma\index{List!comma}
and replaces it with the result. 
If a comma is combined with \verb|@| to produce
the \verb|,@| at-sign,\index{List!at-sign}
then the expression following the at-sign 
is evaluated to create a list. This list is
spliced into place in the template. 
Templates are specially useful for creating
macros, as you will learn below.\index{List!templates}

Macros are programs that brings a more convenient
notation to a standard Lisp form. The syntax of Lisp,
that unifies data and programs, makes it possible to
create powerful macros that implement Domain Specific
Languages (DSLs), speed up coding or create new software
paradigms.\index{Macros!define-syntax}
\index{Macros!syntax-case}\index{datum->syntax}
\index{syntax->datum}
\begin{verbatim}
(define-syntax (while-do stx)
  (syntax-case stx ()
    [ (kwd condition the-return . bdy)
       (datum->syntax #'kwd
	 (let [ (c (syntax->datum #'condition))
		(r (syntax->datum #'the-return))
		(body (syntax->datum #'bdy))]
	   `(do () ((not ,c) ,r) ,@body))) ]))
\end{verbatim}

In the above macro definition,
the  \verb|bdy| variable, which
comes after a dot, groups all
remaining macro parameters.
The syntax of Lisp requires that
a blank space is inserted before
and after the dot.

Nia created an \verb|repl| buffer
to test the \verb|while-do| macro,
as you can see in the following example:\\

\begin{fmpage}{0.8\linewidth}
	\begin{verbatim}
	> (load "macros.scm")
	> (let [ (s '()) (i 0)]
    (while-do (< i 5) s
      (set! s (cons i s))
      (set! i (+ i 1)) ))
      (4 3 2 1 0)
	\end{verbatim}
\end{fmpage}

\begin{fmpage}{0.8\linewidth}
	\verb||
\end{fmpage}

\vspace{0.5cm}
The \verb|(set! i (+ i 1))| operation
destructively updates the value
of the local variable \verb|i|.
For instance, if \verb|i| is
equal to 3, the value of \verb|i|
is replaced with \verb|(+ i 1)|,
what makes \verb|i| equal to 4.
On the same token, \verb|(set! s (cons i s))|
replaces the value of \verb|s| with
\verb|(cons i s)|. Then, if \verb|s= (2 1 0)|
and \verb|i= 3|,
\verb|(set! s (cons i s))| updates \verb|s|
to \verb|(cons 3 '(2 1 0))= '(3 2 1 0)|.

Destructive updates are not
considered good programming practice.
In fact, the \verb|set!| operation has
an exclamation mark to remember you
that it should not be used lightly.
By the way, \verb|set!|
is pronounced as {\em set bang}.

The \verb|while-do| macro is not very useful,
since Scheme programmers adopt the functional
style, which abhor the use of set bang.
A much more interesting macro is the
Curry operator:
\begin{quote}\label{page:lambda2}\index{lambda}
	\begin{verbatim}
	;; (load "macros.scm")

	(define-syntax (while-do stx)
  (syntax-case stx ()
    [  (kwd condition the-return . bdy)
       (datum->syntax #'kwd
	  (let [ (c (syntax->datum #'condition))
		 (r (syntax->datum #'the-return))
		 (body (syntax->datum #'bdy))]
	     `(do () ((not ,c) ,r) ,@body))) ]))

	     (define-syntax (curry stx)
   (syntax-case stx ()
      [ (kwd fun arg)
	(datum->syntax #'kwd
	   (let [ (fn (syntax->datum #'fun))
		  (x  (syntax->datum #'arg))
		  (g  (gensym "var"))]
	      `(lambda(,g) (,fn ,g ,x)) ))] ))
	\end{verbatim}
\end{quote}
In order to understand this macro, you
must learn the concept of lambda abstraction.
You have learned how to define functions
that have names. However,
Lisp alows the creation of
an anonymous functions.
Suppose that you want to filter the
elements of a list, leaving only those
that are greater than a given number.
You can use the following expression:
\begin{quote}
	\begin{verbatim}
	(filter (lambda(x) (> x 4)) '(3 2 4 1 6 9 8))
	(6 9 8)
	\end{verbatim}
\end{quote}
Of course, you could have defined a 
\verb|g4| function and use it as shown below.
\begin{quote}
	\begin{verbatim}
	(define (g4 x) (> 4 x))

	(filter g4 '(3 2 4 1 6 9 8))
	(6 9 8)
	\end{verbatim}
\end{quote}
This solution has a flaw: It requires a
definition for every number that you
want to filter, which is simply impossible.
Fortunately, the lambda abstraction comes
to your rescue, since it 
creates a predicate on the fly for every
number that you want to filter. The Curry
macro is a handier method to define a lambda
abstraction:
\begin{quote}
	\begin{verbatim}
	> (load "macros.scm")
	> (filter (curry > 4) '(3 2 4 1 6 8 9))
	(6 8 9)
	\end{verbatim}
\end{quote}

Perhaps you are not convinced of the
necessity of defining the curry macro.
After all, who  needs to filter a list
for elements greater than a given number?
I will try to give an answer to this
question.

Let us assume that you have a list
of words and needs to sort it
to build a spell checker. The sorting
program is given below.



\begin{figure}[!h]
	\begin{fmpage}{\linewidth}
		\begin{verbatim}
		(define (quick-sort s)
  (cond [(null? s) s]
     [(null? (cdr s)) s]
     [else 
      (append 
	 (quick-sort (filter (curry string<? (car s)) 
	 (cdr s)))
	 (list (car s))
	 (quick-sort (filter (curry string>=? (car s))
	 (cdr s))) )]))
		\end{verbatim}
	\end{fmpage}

	\begin{fmpage}{\linewidth}
		\begin{verbatim}
		> (load "quick.scm")
		> (quick-sort '("Caecilia" "Anna" "Priscilla"))
		("Anna" "Caecilia" "Priscilla")
		\end{verbatim}
	\end{fmpage}
	\caption{The filter function}
	\label{fig:filter}
\end{figure}

In listing~\ref{fig:filter}, the deffinition of
\verb|quick-sort| calls \verb|quick-sort| itself.
When such a thing happens, computer scientists
say that the function is recursive.

It is possible to understand how a recursive
function works by following its execution
step by step. In fact, this has been done
for the \verb|avg| function, in section~\ref{sec:average},
page~\pageref{sec:average}. However, a much
better approach is to study the function logically.

The \verb|append| function appends its
arguments, that are supposed to be lists.
Let us assume that the the \verb|quick-sort|
function was called by the following expression:
\begin{quote}
	\verb|> (quick-sort '("g" "a" "c"  "h" "n" "b"))|
\end{quote}
The \verb|(filter (curry string<? (car s)) (cdr s))|
expression will filter all strings that comes before
\verb|"g"| and return the \verb|("a" "c" "b")|.
A recursive call to \verb|quick-sort| will
sort this list, producing the first argument
of \verb|append|, to wit, \verb|("a" "b" "c")|.

The second argument of \verb|append| is
given by the \verb|(list (car s))| expression.
Since \verb|s= '("g" "a" "c"  "h" "n" "b")|,
and the \verb|list| function builds a list
from its arguments, one
has \verb|(list (car s))= '("g")|.

The third argument of \verb|append| will
be the sorted list of all strings in \verb|s|
that are greater or equal to \verb|"g"|,
i.e., \verb|("h" "n")|.

The value of the \verb|(append '("a" "b" "c") '("g") '("h" "n"))|
expression produces the final result, that is
\verb|("a" "b" "c" "g" "h" "n")|.

\section{Fear of macros}
Now, let us learn how macros work. I myself learned
about macros from a tutorial by Greg Hendershott, and
will try to pass the information ahead in this section.

The first thing you need to learn is that there are
two data types playing cards in a macro definition:
syntax and symbolic expression, or sexpr for short.
An sexpr can be a list, a cons pair, a number, a string,
a symbol or a char. Since you learned Scheme, you know
how to process sexpr with \verb|car|, \verb|cdr| and
\verb|cons|. On the other hand, the machine knows what
to do with syntax.  In order to teach a new trick to the
machine, you need to take a pattern in syntax form, convert
it into sexpr so you can apply transformations on it.
Finally you transform the sexpr into instructions
that the machine already learned, and put the transformed
command back into syntax form. Let us dissect the curry
macro, in order to learn its anatomy.

The \verb|stx| parameter contains the command in the
macro form, that the computer is not able to handle.
The goal of defining a macro is exactly this, teaching
the computer how to deal with the expression that is
stored in the \verb|stx| variable.
\begin{quote}
\begin{verbatim}
(define-syntax (curry stx)
   (syntax-case stx ()
      [ (kwd fun arg)
        (datum->syntax #'kwd
           (let [ (fn (syntax->datum #'fun))
                  (x  (syntax->datum #'arg))
                  (g  (gensym "var"))]
              `(lambda(,g) (,fn ,g ,x)) ))] ))
\end{verbatim}
\end{quote}
Let us recall our goal: We want to perform a syntax
transformation of programs. As Greg Hendershott said,
we could translate syntax into the sexpr form, and
apply the desired transformation using only \verb|car|,
\verb|cdr| and \verb|cons|. However, this is very
tedious. A less error prone strategy is perform pattern
matching. What is a pattern matching? The unification
that you use to process Prolog lists is a kind of pattern
matching. Consider the program below. The Prolog expression
\verb/[H|T]/ is a pattern that selects lists with at least
one element.
\begin{quote}\index{scheme macros!syntax->datum}
	\begin{verbatim}
	%% (load "-f utilities.pl").

	applst([], L, L).
	applst([H|T], L, [H|U]) :- applst(T, L, U).
	\end{verbatim}\index{scheme macros!datum->syntax}
\end{quote}\verb|scheme macros!pattern match|
In the Scheme macro above, \verb|syntax-case| matches
\verb|stx| into a syntax pattern \verb|(kwd fun arg)|.
Then, the definition uses \verb|(syntax->datum #'fun)|
and \verb|(syntax->datum #'arg)| put \verb|arg| and
\verb|fun| into the sexpr form, and perform the
following transformation:
\begin{quote}
\verb|`(lambda(,g) (,fn ,g ,x)) ))|
\end{quote}
Finally, the macro applies \verb|datum->syntax| to the
transformed expression, so that the computer will
understand it. The macro passes \verb|#'kwd| to the
\verb|datum->syntax| in order to provide context for
error messages.

One last thing, \verb|#'kwd| is short for \verb|(syntax kwd)|,
which you can use, if you think that it is clearer.
\index{scheme macros!syntax directive}

\chapter{Recursion}\label{chapter:Recursion}
\index{Recursion}
The mathematician Peano invented a very interesting axiomatic theory for
natural numbers:
\begin{enumerate}\index{Recursion!Peano's axioms}
\item Zero is a natural number.
\item Every natural number has a successor: The successor
      of 0 is 1, the successor of 1 is 2, the successor
      of 2 is 3, and so on.
\item If a property is true for zero and, after assuming
	that it is true for n, you prove that it is true
	for n+1, then it is true for any number.
\end{enumerate}
Peano's insight can be applied to many other situations.
When they asked Myamoto Musashi, the famous Japanese Zen
assassin, how he managed to kill a twelve year old boy
protected by his mother's 37 samurais\footnote{The boy's
father had been killed by Musashi. His uncle met the the
same fate. His mother hired her late husband's students
to protect the child against Musashi.}, he answered:
\begin{quote}\em
I defeated one of them, then I defeated the remaining 36.
To defeat 36, I defeated one of them, then I defeated
the remaining 35. To defeat 35, I defeated one of them,
then\ldots To defeat 2, I defeated one of them, then
I defeated the other.
\end{quote}
A close look will show that the predicate \verb|ap/3|
\index{Recursion!append!definition} of
Listing~\ref{rec/ap} acts like Musashi. The first
clause of \verb|ap([],L,L)| states that appending
an \verb|[]| empty list to \verb|L| produces \verb|L|.
The second  clause of \verb|ap/3|  states:
\begin{quote}
\begin{verbatim}
ap([H|T],L,[H|U]) :- ap(T,L,U).
   %% The result of appending [H|T] to L is [H|U],
   %% if the result of appending T to L is U.
\end{verbatim}
\end{quote}

In listing~\ref{rec/ap}, the second clause of the \verb|ap/3|
predicate changes the goal repeatedly from
\verb/ap([H|T],L,[H|U])/ to \verb/:- ap(T,L,U)/, until a new
subgoal unifies with the first clause and produces a solution.
By the way, the \verb|:-| symbol can be read as {\em if},
thus the clause \verb/ap([H|T],L,[H|U]) :- ap(T, L, U)/ means:
\begin{quote}
\verb/app([H|T],L,[H|U])/ if \verb/ap(T,L,U)/
\end{quote}

Let us pick a more concrete instance of the problem.
Nia evaluated \verb|?- ap([2,3], [4,5],R).| Here are the
inference steps to solve this query:
\begin{enumerate}
\item The query \verb/?- ap([2,3],[4,5],L)/ matches
to the \verb/ap([H|T],L,[H|U])/ head of the second clause,
with \verb|H=2|, \verb|T=[3]| and \verb|T=[4,5]|.
The second clause changes the goal to:
\begin{quote}
	subgoal 1 -- \verb/:-ap([3],[4,5],U)/
\end{quote}
after eventually finding that the subgoal 1 produces
\verb|U=[3,4,5]|, the inference engine will mount
\verb/H=2/ and \verb/U=[3,4,5]/ into the \verb/[H|U]/
pattern to obtain the final result: \verb/[2,3,4,5]/.
\item Subgoal 1 -- which is \verb/:-ap([3],[4,5],U)/
      -- fails to unify with the first clause, but
      matches the second clause with \verb|H=3|,
      \verb|T=[]| and \verb|L=[4,5]|. The second clause
      changes the goal again, this time to
\begin{quote}
      subgoal 2-- \verb|:-ap([],[4,5],U)|.
\end{quote}
Subgoal 2 unifies with the first clause,
and produces \verb/U=[4,5]/. Then, the inference
engine mount \verb|H=3| and \verb|U=[4,5]| into
the \verb/[H|U]/ pattern of the second clause,
and thus solves subgoal 1 --\verb/ap([3],[4,5],[3,4,5])/.
\item Now that the solution of subgoal 1 is known,
the answer to the original query can be found simply
by mounting \verb/H=2/ and \verb/U=[3,4,5]/ into
the pattern \verb/[H|U]/, which produces \verb/[2,3,4,5]/
\end{enumerate}

\begin{figure}[!h]
\begin{fmpage}{0.8\linewidth}
\begin{verbatim}
ap([], L, L).
ap([H|T], L, [H|U]) :- ap(T, L, U).
\end{verbatim}
\end{fmpage}

\begin{fmpage}{0.8\linewidth}
\begin{verbatim}
> ?- ap([1,2,3],[4,5],Resp).
Resp = [1,2,3,4,5]
\end{verbatim}
\end{fmpage}
\caption{Append}
\label{rec/ap}
\end{figure}


\section{Classifying rewrite rules}\index{Recursion!Classifying rules}
Typically a recursive definition has two kinds of clauses:
\begin{enumerate}
\item Trivial cases, which can be \index{Recursion!Trivial case} 
resolved using primitive operations and unification, which is
a kind of pattern match.
\item General cases, which can be \index{Recursion!General case} 
broken down into simpler subgoals.
\end{enumerate}
Let us classify the two clauses of \verb|ap/3|:\\

\verb||\\
\begin{tabular}{p{5cm}p{7cm}}
\verb|ap([],L,L) | & The first clause is certainly trivial\\
\\
\verb/ap([H|T],L,[H|U])/ \verb/:- /
\verb/    ap(T, L, U)/
& The second clause can be
broken down into simpler
subgoals and operations: Appending two
lists with one element
removed from the first,
and then inserting the element
left out into the result.\\
\end{tabular}


\section{Quick Sort}
Although Hoare's Quick Sort algorithm is
of little practical use in these days of BTree everywhere, it gives a good illustration
of recursion. The problem that Hoare solved consists of sorting a list.

The \verb|(smaller xs p)| function returns
all elements of \verb|s| that are smaller
than the \verb|p| pivot. 
The \verb|(greater xs p)| produces the
list of the \verb|s| elements that are
greater or equal to \verb|p|. Therefore the
\verb|(quick xs)| function partitions 
\verb|xs| into three lists, then sorts
and appends these lists:
\begin{itemize}
\item \verb|(smaller (cdr s) (car s))|--
numbers smaller than \verb|(car s)|
\item \verb|(list (car s))| 
\item \verb|(greater (cdr s) (car s))|--
numbers greater or equal to \verb|(car s)|.
\end{itemize}
If you sort, then concatenate these three lists,
you will end up sorting the original list.
A concrete case will make this fact clear.
Let \verb|s| be the list \verb|'(4 3 1 5 2 8 7)|.
The expression
\verb|(quick (smaller (cdr s) (car s)))|
returns \verb|'(1 2 3)|.
The expression
\verb|(quick (greater (cdr s) (car s)))|
generates \verb|'(5 7 8)|. Finally, 
the 
\verb|(append '(1 2 3) '(4) '(5 7 8))|
application
returns \verb|'(1 2 3 4 5 7 8)|.

In the quicksort algorithm,
there are two trivial cases,
the empty \verb|'()| list
and lists with a single element,
either of which
do not require sorting.

Lists with two or more elements
are sorted by breaking them into
smaller sublists. Since they are
closer to the trivial cases,
one can assume that the smaller sublists
are easier to sort.

Figure~\ref{quicksort} shows a complete
implementation of the quicksort algorithm
for a list of numbers.

\begin{figure}[!h]
\begin{fmpage}{0.9\linewidth}
\begin{verbatim}
;; (load "quicksort.scm")


;; Output: elements of xs that are smaller than p

(define (smaller xs p)
  (cond [ (null? xs) xs]
        [ (< (car xs) p]
          (cons (car xs)
                (smaller (cdr xs) p)))
        [else (smaller (cdr xs) p)]))


;; Output: elements of xs greater or equal to p

(define (greater xs p)
   (cond [ (null? xs) xs]
         [ (>= (car xs) p)
           (cons (car xs)
                 (greater (cdr xs) p))]
         [else (greater (cdr xs) p)] )) 

(define (quick s)
  (cond [ (null? s) s]
        [ (null? (cdr s)) s]
        [else (append 
                  (quick (smaller (cdr s) (car s)))
                  (list (car s))
                  (quick (greater (cdr s)
                                  (car s))) )] ))

\end{verbatim}
\end{fmpage}

\begin{fmpage}{0.9\linewidth}
\begin{verbatim}
> (quick '(4 3 1 5 2 8 7))
(1 2 3 4 5 7 8)
\end{verbatim}
\end{fmpage}
\caption{The quicksort algorithm}
\label{quicksort}
\end{figure}

\section{Named-let}

An efficient way to implement repetition is through auxiliary
parameters, that avoids recursive calls nested in other
functions.\\

\begin{fmpage}{0.9\linewidth}
\begin{verbatim}
(define (fn+1 n fn fn-1)
  (if (< n 2) fn
      (fn+1 (- n 1) (+ fn fn-1) fn)) )
\end{verbatim}
\end{fmpage}

\begin{fmpage}{0.9\linewidth}
\begin{verbatim}
> (fn+1 5 2 1)
13
\end{verbatim}
\end{fmpage}

\verb||\\
The above definition has the
undesirable characteristic of
requiring two auxiliary arguments,
and users cannot forget to initialize
these dummy arguments. 

The named-let\index{Named-let}\index{Loop!named-let}
permit the creation of functions
with initialized arguments, avoiding
auxiliary parameters that require manual initialization.

\begin{figure}[!h]
\begin{fmpage}{0.9\linewidth}
\begin{verbatim}
(define (fibo i)
(let fn+1 ( (n i) (fn 1) (fn-1 1) )
  (if (< n 2) fn
      (fn+1 (- n 1) (+ fn fn-1) fn)) ))

\end{verbatim}
\end{fmpage}

\begin{fmpage}{0.9\linewidth}
\begin{verbatim}
> (load "Fibonacci.scm")
> (fibo 5)
8
\end{verbatim}
\end{fmpage}
\caption{One argument Fibonacci function}
\label{oneargfib}
\end{figure}

The named-let is used mainly
to loop. Therefore, it often replaces looping
facilities that one can find in languages
like C. However, the named-let binding has a
clear advantage over the competition: One can
give a meaningful 
name\index{Named-let!meaningful name for loop}
to scheme loops.
For instance, since the loop of
figure~\ref{oneargfib} calculates the \verb|fn+1|
iteration, Nia put the \verb|fn+1| tag on it.
Racket has another way to get rid of auxiliary 
parameters:\index{Local variables!auxiliary parameters}
arguments with default values, as shown below. Unfortunately,
default values are not portable.

\verb||\\
\begin{fmpage}{\linewidth}
\begin{verbatim}
(define (avg s sum  n)
  (cond [ (and (null? s) (= n 0)) 0]
    [ (null? s) (/ sum n) ]
    [else (avg (cdr s) (+ (car s) sum) (+ n 1.0))] ))
\end{verbatim}
\end{fmpage} \index{Local variables!default values}\\
\noindent
\begin{fmpage}{\linewidth}
\begin{verbatim}
> (avg '(3 4 5 6) 0 0)
4.5
\end{verbatim}
\end{fmpage}

\section{Reading data from files}\index{File}

Figure \ref{rdFile} reads and prints a file line by line.
However, before printing a string, \verb|(port->lines p)|
prefixes it with a commented line number.

\begin{figure}[!h]
\begin{fmpage}{\linewidth}
\begin{verbatim}
;; Racket: uncomment the three lines below
;#lang racket
;(provide rdLines)
;(define get-line read-line)

(define (port->lines p)
   (let next-line ( (i 1) (x (get-line p)) )
     (cond [ (eof-object? x) #t]
        [else (display "#| ")
              (display (number->string i))
              (cond [ (< i 10)  (display "  |# ")]
                    [else  (display " |# ")])
              (display x) (newline)
              (next-line (+ i 1) (get-line p))] ))) 

(define (rdLines filename)
   (call-with-input-file filename port->lines))
\end{verbatim}
\end{fmpage}

\begin{fmpage}{\linewidth}
\begin{verbatim}
> (load "prtFile.scm")
> (rdLines "average.scm")
#| 1  |# (define (avg xs)
#| 2  |#   (let nxt [(s xs) (acc 0) (n 0)]
#| 3  |#     (cond [ (and (null? s) (= n 0)) 0]
#| 4  |#       [ (null? s) (/ acc n) ]
#| 5  |#       [else (nxt (cdr s) (+ (car s) acc)
#| 6  |#                  (+ n 1.0))] )))
\end{verbatim}
\end{fmpage}
\caption{Lines from file}
\label{rdFile}
\end{figure}

Lisp has a cleverly designed input system.
The \index{call-with-input-file} \index{Input from file}
\verb|call-with-input-file| \label{page:call-with-input-file}
procedure has two arguments. The first
argument is the file name. 
The second argument is
a single parameter function such
as \verb|port->lines|. In the example
of figure~\ref{rdFile}, Lisp 
opens a file, and pass the file descriptor
to the \verb|port->lines| procedure.

The definition of \verb|(port->lines p)| assigns
a line that it reads  from port \verb|p| to the
variable \verb|x|. Then \verb|x| is printed, and
\verb|loop| proceeds to read the next line. The
iteration stops when  end of file is reached.

\section{Writing data to files}
Let us assume that Nia needs 
a Scheme program that translates
markdown to html. She wants to
use the program to publish poems
in the internet. Here an example
of markdown:
\begin{quote}
\begin{verbatim}
# To Helen
## By Edgard Alan Poe

Helen, thy beauty is to me
Like those Necéan barks of yore,
That gently, o'er a perfumed sea,
The weary, way-worn wanderer bore
To his own native shore.

On desperate seas long wont to roam,
Thy hyacinth hair, thy classic face,
Thy Naiad airs have brought me home
To the glory that was Greece,
And the grandeur that was Rome.

Lo! in yon brilliant window-niche 
How statue-like I see thee stand, 
The agate lamp within thy hand! 
Ah, Psyche, from the regions which 
Are Holy-Land! 
\end{verbatim}
\end{quote}

The program of listing~\ref{wrtFile}
reads a markdown file line by line.
If the line starts with the \verb|##| prefix,
it will be wrapped in the \verb|<h1>...</h1>|
html tag. If its first char is the \verb|#\#|
prefix, but the second char is different
from \verb|#\#|, it will be wrapped in
the \verb|<h2>...</h2>| html tag.

The simplified version of the html
generator of listing~\ref{wrtFile} treats
only titles, line breaks and paragraphs.
However, the interested reader will
be able to add other html elements.

You have already learned how the
procedure \verb|call-with-input-file| works.
If you don't remember, take a look at
page~\pageref{page:call-with-input-file}.

The procedure\index{call-with-output-file}
\verb|call-with-output-file|\index{Output to file}
can be\index{Output to file!call-with-output-file}
applied to a file name and a function
with an output port as argument. One could
define the output port function, but
usually people use a lambda abstraction
to perform the magic. Read again the
explanation about the lambda abstraction
on pages~\pageref{page:lambda1}
and~\pageref{page:lambda2}. File input/output
must be mastered to perfection.

\begin{figure}[!h]
\begin{fmpage}{\linewidth}
\begin{verbatim}
;;Racket: uncomment the three lines below
;#lang racket
;(provide copyFile)
;(define get-line read-line)

(define (tag i <h> x </h> )
   (let [ (Len (string-length x))]
      (string-append <h>
          (substring x i (- Len 1)) </h> "\n")))

(define (subtitle? x)
  (and (> (string-length x) 4)
       (char=? (string-ref x 0) #\#)
       (char=? (string-ref x 1) #\#)))

(define (title? x)
   (and (> (string-length x) 3)
        (char=? (string-ref x 0) #\#)))

(define (convert in out)
  (let loop ( (x (get-line in)) )
     (cond [ (eof-object? x) #t]
        [ (subtitle? x)
          (display (tag 2 "<h2>" x "</h2>") out)
          (loop (get-line in))]
        [ (title? x)
          (display (tag 1 "<h1>" x "</h1>") out)
	  (loop (get-line in))]
        [ (> (string-length x) 1)
          (display x out)
          (display "<br/>\n" out)
          (loop (get-line in))]
        [ else (display "<p/>\n" out)
	    (loop (get-line in))] )))

(define (copyFile inFile outFile)
   (call-with-output-file outFile
      (lambda(out) (call-with-input-file inFile
         (lambda(in) (convert in out))) )))
\end{verbatim}
\end{fmpage}

\begin{fmpage}{\linewidth}
\begin{verbatim}
> (load "tohtml.scm")
> (copyFile "readme.md" "readme.html")
#t
\end{verbatim}
\end{fmpage}
\caption{Writing data to a file}
\label{wrtFile}
\end{figure}

\begin{thebibliography}{99}


\bibitem{Abelson} Harold Abelson, Gerald Jay Sussman and
Julie Sussman. Structure and Interpretation of
Computer Programs. The MIT Press, Second Edition, 1996.

\bibitem{Felleisen} Matthias Felleisen,
Robert Bruce Findler, Matthew Flatt, Shriram Krishnamurthi.
How to Design Programs.

\bibitem{Sitaram} Dorai Sitaram. Teach yourself Scheme
in Fixnum Days. Available
from \verb|http://ds26gte.github.io/tyscheme/|

\bibitem{Hoyte} Doug Hoyte. Let Over Lambda. Hoytech, 2008. ISBN: 978-1-4357-1275-1

\bibitem{Church1932} A. Church, A set of postulates for the foundation of logic, Annals of Mathematics, Series 2, 33:346–366 (1932).

\bibitem{Church} Alonzo Church. The Calculi of Lambda Conversion. Princepton University Press, 1986.

\bibitem{Dybvig} R. Kent Dybvig. The Scheme Programming Language.
Available from \verb|http://www.scheme.com/tspl4/|

\end{thebibliography}

\printindex

\end{document}

